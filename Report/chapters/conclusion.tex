\chapter{Conclusion}

In this master thesis, we have analyzed how existing tools support the testing and debugging of cross-device web applications. We found that there are many tools available for testing traditional web applications and also responsive web applications, but all those tools have significant disadvantages when it comes to cross-device application testing. There are some significant differences between cross-device applications and traditional web applications: The biggest difference is that multiple different devices that show different content are typically involved in a cross-device scenario. Those differences make it difficult to debug cross-device applications using tools aimed at testing traditional and responsive applications that usually only involve one device at a time. We also analyzed existing cross-device application development frameworks and found that they provide little to no support for testing and especially debugging the applications. However, many of those frameworks would benefit from a tool that helps debugging the applications developed with them.

With the limitations discovered during our analysis in mind, we specified a set of requirement that a cross-device application testing and debugging tool should fulfill. Using those requirements, we implemented XDTools, a set of tools that facilitates the testing and debugging of cross-device applications. XDTools is implemented as a web application and is thus largely independent of the system it is running on. XDTools includes concepts that have already been seen in responsive web application testing tools as well as concepts that are well-known from browser debugging tools and only required extension to suit the needs of cross-device applications. XDTools also features a record and replay tool. Record and replay has already proven to be useful for reproducing bugs in traditional web applications and it provides additional value for cross-device applications where multiple users can be simulated with such a tool. Finally, our system also includes some features targeted specifically at cross-device applications, such as an automatic connection tool. 

Using XDTools, we also implemented two sample applications that provided useful insights for further improvements to XDTools and also served as basis for our user study. 

Finally, we conducted a user study to evaluate our system. The participants of our study had to complete multiple tasks where they had to fix a bug or implement a feature, either using XDTools or only using traditional browser debugging tools. The results of the user study show that XDTools indeed helps with debugging cross-device applications and we received some enthusiastic feedback from participants that already had some previous experience with cross-device applications and struggled with testing them. 

Overall, we think that XDTools provides a number of useful features for cross-device application testing and debugging and adds a lot of value to the existing web-based cross-device frameworks. Testing cross-device applications is an important step on the way to a more widespread adaptation of cross-device applications and we hope that it will help advance the field further.

\section{Future Work}

We will now present a few ideas for future work that could improve XDTools.

\subsection{Extended Record and Replay}

So far, record and replay only records user interactions. However, there are a lot of other factors that influence the behavior of a web application. In the future, record and replay could be extended to support recording of other non-deterministic factors. Those factors could for example include:
\begin{itemize}
	\item Dates
	\item Random numbers
	\item Server responses
	\item Interrupts by \lstinline|setTimeout| and \lstinline|setInterval|
\end{itemize}
Recording additional data could lead to more accurate replaying of events and could help reproduce some bugs that could not be reproduced so far. Furthermore, the replaying of events could be improved: So far, if data is recorded on one device and then replayed on another device, the coordinates of things like click events are kept. The same applies for scrolling where the scrolling position is set manually. However, if devices have different resolutions, this could lead to inaccurate replaying of events. Ideally, the events would be adjusted to the target resolution before replaying.

\subsection{Extended Device Emulation}

The device emulation in XDTools only emulates the resolution of the devices. However, the resolution is not the only difference between a desktop device and a mobile device. Other aspects could be considered when emulating devices for a more realistic experience:
\begin{itemize}
	\item Touch interactions
	\item Network conditions
	\item Location
	\item Input from sensors
	\item Hardware performance
\end{itemize}
Although an emulated device will never be completely equivalent to a real device, including more factors into device emulation can enable more sophisticated testing on emulated devices. Emulating aspects like location and network conditions could even reveal issues that are not necessarily found when testing on real devices. Testing on real devices typically happens inside the office, a somewhat artificial setting. In the office, network conditions will typically be good and sensor readings like location do not change much. If those things can be emulated, testing applications in the office can become more realistic and useful.

\subsection{Tighter Integration with Browser}

It would be very useful if XDTools could be integrated tighter with the debugging tools provided by browsers and also with the browser itself. Even though we actually implemented a Chrome DevTools extension, the available APIs only allow rather limited access to the debugging tools. Tighter integration with the debugging tools could further improve the debugging capabilities, especially concerning JavaScript debugging: Within XDTools, it was only possible to add breakpoints at the beginning of functions and functions that are not globally accessible cannot be debugged at all. In the ideal case, it would be possible to set a breakpoint on one device and automatically transfer it to other devices if desired. 

Furthermore, XDTools' shared JavaScript console and CSS editor provide less features than the console and CSS editor included directly in the browser. If our aggregation features would be integrated directly into the browser debugging tools, more extensive debugging would be possible.

If XDTools would be integrated directly into the browser, the DNS server would not be required anymore. Instead, the browser could just make sure that different devices do not share local resources. With everything from XDTools included directly into the browser, no installation of additional components would be required anymore and debugging cross-device applications would be possible directly from the browser's debugging tools.

\subsection{Long-Term Study}

Although our user study yielded promising results, it was a clearly artificial setting: The tasks were rather short and the participants were not familiar with the application they worked with. Thus, they also spent some getting to know the application and understanding the code. Also, the participants only had access to a subset of the code of the entire application. In the real world, developers typically deal with a much larger amount of code. Additionally, the instructor was always sitting next to the participants and a camera was recording the participants. This might influence the behavior of the participants. Finally, some features were disabled for the user study and could not be evaluated yet.

It would be very useful if a long-term study could be conducted to see how developers use XDTools in their everyday jobs. Giving developers that are in the process of developing a cross-device application access to XDTools during the entire course of development would help further evaluate XDTools. The developers would be in their natural ``habitat'' and could use XDTools in a way that best fits their needs. The developers could provide feedback during the development of the application and the actual usage of XDTools could be recorded and analyzed to gain further insights. After finishing development of their application, developers could be interviewed, giving them a chance to tell us what they think of XDTools and provide suggestions for further improvements. Conducting such a long-term user study could be very helpful when releasing future versions of XDTools. As a first step for further evaluation and improvement of XDTools, we plan to make XDTools available on Github\footnote{\url{https://github.com/}} in the near future. Hopefully, some developers can already benefit from XDTools if it is available to the public and we might also receive some valuable feedback.