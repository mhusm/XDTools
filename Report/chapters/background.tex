\chapter{Background and Related Work}

In the following sections, we will present some background information and related work for our project. We will start by analyzing existing tools for debugging traditional web applications, including tools that are integrated directly into browsers. We will also discuss some tools aimed at testing responsive web applications. Finally, we will present a few cross-device application development frameworks that either provide some cross-device testing facilities themselves or that could benefit from such facilities.

\section{Web Application Testing}

In the following subsections, we will present a few tools supporting the testing and debugging of traditional web applications. Those tools are also relevant for cross-device applications, as most cross-device applications are web applications.

\subsection{Browser-Integrated Debugging Tools}

Over the last few years, browsers have added more and more tools for debugging web applications. Chrome probably provides the most advanced tools, but all browsers include at least some tools for debugging web applications. We will now present a few of those tools provided by browsers themselves.

\subsubsection{Chrome DevTools}

Google Chrome already provides a wide range of features for testing and debugging web applications. First of all, it lets the developer inspect the DOM tree  and allows inspection and on-the-fly editing of DOM elements. Furthermore, new CSS rules or properties can be added or existing ones can be modified. Another useful feature is the JavaScript Console. It has two main purposes: First, it can be used to log diagnostic information in the development process. Second, it is a shell prompt which can be used to interact with the document and DevTools. Chrome DevTools can also be used to debug JavaScript. In the DevTools, all scripts that are part of the inspected page can be seen. Breakpoints can be set in the scripts and standard controls to pause, resume and step through code are provided. All features mentioned before are also available for debugging remote devices. Remote debugging can be used by connecting a device to the desktop PC with a cable.

Chrome DevTools provides many more features useful for debugging, but explaining them all would exceed the scope of this document. While all those features are very useful for testing and debugging web applications, the main disadvantage is that they can only be used to debug one device and one website at a time.

\subsubsection{Firefox Developer Tools}

Firefox\footnote{\url{https://developer.mozilla.org/en-US/docs/Tools}} provides developer tools similar to Chrome DevTools. It also has a page inspector that allows developer to inspect the DOM tree and modify the CSS. It also features a console, a JavaScript debugger and remote debugging. Like Google Chrome's DevTools, Firefox's developer tools provides more features that will not be explained here. Furthermore, Firefox gives developers access to a feature called Developer Toolbar. The Developer Toolbar is a command-line tool that can be used to access a number of developer tools from within Firefox. Overall, Chrome DevTools and Firefox Developer Tools are very similar and thus also have the same limitations. Just like Google Chrome, only one device can be debugged at a time in Firefox.

\subsubsection{Summary}

Internet Explorer, Microsoft Edge, Opera, Safari and other browsers all provide some kind of developer tools. However, they do not provide any features that we did not already mention before, thus we will not provide any detailed description of the developer tools available in those browsers. All those browsers share the same limitation: Their tools can only be used to debug one device at a time. For cross-device applications however, it would be desirable to debug multiple devices at a time without having to navigate between windows all the time.

\subsection{Record and Replay}

Record and replay has already successfully been used for debugging web applications. It allows developers to quickly reproduce bugs that are otherwise difficult to reproduce. We will investigate a few tools that support record and replay in web applications.

\subsubsection{Timelapse}

In~\cite{timelapse2013}, Burg et al. describe their record and replay tool Timelapse. Timelapse is a tool for recording, reproducing, and debugging interactive behaviors in web applications. Timelapse is built on Dolos, a record/replay infrastructure that ensures deterministic execution by capturing and reusing user inputs, network responses, and other nondeterministic inputs. Developers can use Timelapse to browser, visualize, and seek within recorded program executions while simultaneously using familiar debugging tools such as breakpoints and logging. The developers of Timelapse conducted a user study with 14 web developers which showed no significant effect on task times, task success, or time spent reproducing behaviors when developers had access to Timelapse. Expert developers seemed to better integrate Timelapse into their workflow, using Timelapse to accelerate familiar tasks rather than redesigning their workflow. However, Timelapse distracted less-skilled developers that were led astray by unverified assumptions.

\subsubsection{Mugshot}

Mugshot, developed by Mickens et al.~\cite{mugshot2010}, is a record/replay system that captures all events in a JavaScript program, allowing developers to deterministically replay past executions of web applications. The goal of Mugshot is to provide low-overhead, "always-on" capture and replay for web-deployed JavaScript programs. Mugshot logs explicit user interactions like mouse clicks as well as background activities such as random number generation and the firing of timer callbacks. The client-side log is sent to the developer in response to a trigger like an unexpected exception being caught. The developer can then use Mugshot's replay mode to recreate the original JavaScript execution on his unmodified browser. 

\subsubsection{WaRR}

WaRR~\cite{warr2011} is a high-fidelity, "always-on" tool that records and replays the interactions between users and web applications. The WaRR recorder is embedded directly into the web browser and the WaRR Replayer uses an enhanced, developer-specific web browser that enables more realistic simulation of user interactions based on the recorded traces. Thus, the recording functionality is an integral part of the browser. Andrica et al. developed two tools on top of WaRR:
\begin{itemize}
	\item WebErr allows testing of web applications against human errors, i.e. navigation errors and timing errors.
	\item AUsER automatically generates user experience reports: If a user experiences a bug while using a web application, they press a button and the developers of the application receive the sequence of actions that led to the bug.
\end{itemize}
Using WebErr, the developers of WaRR were able to find a bug in Google Sites\footnote{\url{https://www.google.com/work/apps/business/products/sites/}}.

\subsubsection{Rumadai}

Yildiz et al.~\cite{rumadai2012} developed Rumadai, a Visual Studio plug-in that helps developers test web applications by recording and replaying client-side events. Rumadai injects JavaScript code into web pages to be deployed at servers. The injected code records user events as well as client-side dynamic content requests and their responses. The recorded events are sent to a database and can be queried by the developers of the web page. The recorded events can then be replayed in a browser using Rumadai seamlessly from Visual Studio.

\subsubsection{FireCrystal}

FireCrystal~\cite{firecrystal2009} is a Firefox extension that allows developers to extract the implementation details of interactive behaviors from other websites. Developers can tell FireCrystal to start recording and demonstrate the interactive behavior they want to extract. FireCrystal records the interaction, keeping track of DOM changes, JavaScript executions and user input events. The developer can then replay the interactions and FireCrystal displays the HTML, CSS and JavaScript code that affected a particular element at any specific time. FireCrystal also provides an execution timeline that developers can scrub back and forth.

\subsubsection{Summary}

The tools described above all employ record and replay in some way for improving web applications. FireCrystal focuses more on extracting interactive behaviors from other websites, but it also provides basic record and replay mechanisms. The other tools all focus on debugging web applications. WaRR and Mugshot provide recording mechanisms directly to users of web applications, providing them with means of submitting bug reports to the developers of the applications. Some of those tools are designed for only replaying event sequences on the devices they were recorded, while others allow replaying event sequences on different devices, e.g. the sequences can be replayed on the developer's machine after a user has recorded the bug. However, all of those tools have one limitation in common: They focus on replaying an event sequence on one device at a time. 

Record and replay mechanisms have already been shown to help with debugging in traditional web applications, thus they could certainly also be useful for debugging cross-device applications. In particular, we believe that replaying event sequences on multiple devices simultaneously could help simulating multiple users using a cross-device application. Developers usually work alone when fixing a bug, or in very small groups, which makes reproducing bugs that require multiple devices to interact simultaneously a very difficult task. If multiple devices should replay event sequences simultaneously, mechanisms for accurately timing event execution are also required. Most of the tools described above do not include such mechanisms, or they include some mechanisms that are however not sufficient for configuring timing in a cross-device replaying scenario.

\section{Responsive Web Application Testing}

There are a number of tools for testing responsive websites. Some of them can be accessed on the web as a service, while others are desktop programs that have to be installed. Some focus more on testing on real devices and others focus on emulating devices. In the following subsections we will describe some of those tools. Responsive web applications have a few similarities with cross-device applications: Both types of applications are supposed to work on a number of different devices with different screen sizes, input capabilities and more. Thus, tools for testing responsive web applications are to some extent also suited for testing cross-device applications.

\subsection{Web Services}

The following two tools can be accessed through a browser and do not have to be installed.

\subsubsection{BrowserStack}

BrowserStack\footnote{\url{https://www.browserstack.com/}} allows developers to select browsers and devices and then generate screenshots. It is also possible to live test one device at a time. Furthermore, there are developer tools for remote devices and Selenium cloud testing is possible. Advantages of BrowserStack are that real iOS devices are used for screenshots and interactions can be tested automatically using Selenium. A major disadvantage is that only emulated Android devices are available, which limits the usefulness of the tool. Furthermore, only one device can be live tested at a time which is a huge disadvantage for cross-device application testing. 

\subsubsection{CrossBrowserTesting}

With CrossBrowserTesting, developers can select a number of devices as well as the operating system, browser and resolution and generate screenshots. The layout differences between different devices can then automatically be analyzed. Furthermore, websites can also be live tested and Selenium automated testing is available as well. The main advantage of CrossBrowserTesting is that all screenshots are generated on real devices and a very wide variety of devices is available. Also, interactions can be tested. The usefulness of the tool is again limited by the fact that live testing is only possible on one device at once. Additionally, while detecting layout differences is a useful feature in general, it is not yet very mature and some layout differences that are detected seem rather trivial (the body element of a larger device is larger), while other differences are not noticed at all.

\subsection{Testing with Actual Devices}

Using the following tools, developers can connect and debug real devices. 

\subsubsection{Remote Preview}

Remote Preview\footnote{\url{https://github.com/viljamis/Remote-Preview}} allows synchronizing URLs across multiple devices. This allows fast previewing of a website on multiple devices. In cross-device scenarios, it may provide especially useful for quick connecting of devices in applications where devices are connected simply by copying the same URL to all devices. However, the fact that this tool only provides one feature, namely URL synchronizing, limits its usefulness.

\subsubsection{BrowserSync}

BrowserSync provides a large number of features for testing websites. It allows remote debugging of HTML and CSS, can add CSS outlines or box shadows to all elements and add a CSS grid overlay. It can also load a URL on all devices or refresh all devices as well as automatically refresh devices when files are changed. Furthermore, it allows synchronizing interaction between devices, i.e. clicks, scrolls, form submits, form inputs and form toggles. It also provides network throttling. The advantages of BrowserSync are that it provides a wide range of features, including synchronizing interactions, which is a feature that distinguishes it from other tools. However, for cross-device application testing, synchronizing interactions among all devices is of limited usefulness, as different devices have different roles and thus also different responsibilities.

\subsubsection{Adobe Edge Inspect CC}

Adobe Edge Inspect CC allows developers to take screenshots on all connected devices simultaneously. The screenshots are then automatically transferred to a folder on the desktop PC. It can also refresh all devices simultaneously. Furthermore, the URL that is opened on the desktop PC is loaded on all other connected devices. It also allows remote HTML and CSS debugging using weinre. The main advantage that distinguishes Adobe Edge Inspect CC from other tools is that is simply synchronizes the URL that the developer is currently looking at on the desktop PC, thus it works even if the developer switches tabs or browser windows. However, the fact that the URL cannot be changed from devices other than the desktop PC could be a disadvantage in some scenarios. Also, the installation process is rather extensive: A program needs to be installed on the desktop PC as well as a Chrome extension and an app needs to be installed on all mobile devices that the developer wants to connect. Regarding cross-device applications, again, refreshing all devices at once can be useful as well as synchronizing URLs in some cases, but other than that, it does not provide any features that help with cross-device application testing.

\subsubsection{Ghostlab}

Ghostlab\footnote{\url{http://www.vanamco.com/ghostlab/}} is one of the more mature tools for website testing and provides features similar to Browser Sync. It can also load a URL on all devices or refresh all devices at once and refresh automatically when files are changed. It provides means for synchronized browsing as well as synchronized HTML and CSS inspection on multiple devices. Furthermore, it can automatically fill out forms and provides remote Javascript debugging. Synchronization can be turned on and off on a per-device basis, which makes the tool more useful than tools that simply synchronize all devices. However, the usefulness is still limited because constantly changing the devices that should be synchronized is time-consuming and error-prone. Also, interactions on cross-device applications typically do not happen in a synchronous fashion.

\subsection{Device Emulation}

The following tools allow developers to emulate devices. 

\subsubsection{Chrome Device Mode}

Google Chrome provides extensive support for emulating devices with its Device Mode. If the developer opens the DevTools, they can switch to the Device Mode and emulate any device. The developer can select an arbitrary device from a large list of predefined devices or create a custom device. They can also enable network throttling, touch emulation and location emulation. The large number of different aspects that are emulated in Device Mode make it very well suited for testing responsive web applications. While most other device emulation tools are limited to emulating resolution and maybe touch, Chrome Device Mode allows to emulate many other aspects as well.

\subsubsection{Firefox's Responsive Design View}

Firefox's Responsive Design View also allows developers to emulate devices, similar to Chrome's Device Mode, but it provides fewer features. Instead of focusing on existing devices, it provides a short list of common resolutions that can be selected by the developer. It also allows to emulate a custom resolution. Additionally, the orientation of the devices can be switched, touch can be emulated, and screenshots of the displayed web page can be taken.

\subsection{Summary}

While the tools described above already provide a wide variety of features that are immensely useful for responsive web application testing as well as web application testing in general, the distinguishing characteristics of cross-device applications lead to a limited usefulness of those tools. First of all, there are two different approaches to live testing in the tools described above. The first is to test on real devices, but through a web service. In those tools, live testing is possible on one device at a time, but in cross-device scenarios, multiple devices are typically involved. The second approach is to let the developer connect their own devices and synchronize interactions among all or some devices. This is also problematic for testing cross-device applications because not all devices perform the same interactions and those that do, do not necessarily perform them at exactly the same time. Two features that some of the tools mentioned above provide and that would definitely also be useful for cross-device applications are refreshing all devices at once and loading a URL on all devices. Remote HTML, CSS and JavaScript debugging are also desirable in a cross-device scenario, but this is already covered by Google Chrome anyways. Something similar to Selenium testing would clearly also be useful in cross-device scenarios. However, depending on the device, different tests would be needed and it should be possible to run those tests in parallel.

\section{Cross-Device Application Development Frameworks}

In the following subsections, we will describe some frameworks that facilitate the development of cross-device applications. Some of those frameworks already provide some mechanisms for testing the applications developed with them, but other include none and could benefit a lot from tools that would allows easy testing of cross-device applications.

\subsection{Frameworks that have some Testing Tools}

The following frameworks already include some mechanisms for testing and debugging applications, but most of those mechanisms are limited to specific aspects of the applications and require the applications to be developed with the framework.

\subsubsection{XDStudio}

In ~\cite{xdstudio2014}, Nebeling et al. present their web-based GUI builder, XDStudio. XDStudio is designed to support interactive development of cross-device web interfaces. It has two complementary authoring modes: \emph{Simulated authoring} allows designing for a multi-device environment on a single device by simulating other target devices. \emph{On-device authoring} allows the design process itself to be distributed over multiple devices. The design process is still coordinated by a main device, but directly involves target devices. The user can switch between two different modes: In \emph{use mode}, the user can interact normally with the interface loaded into the editor. In \emph{design mode}, the user can manipulate the interface directly. XDStudio also allows the specification of \emph{distribution profiles} in terms of involved devices, users and target user interfaces.

Although XDStudio includes mechanisms for inspecting interactive designs on emulated devices and on connected real devices, there is no specific support for debugging such as access to the console. Thus, XDStudio is well suited for seeing what an application looks and feels like on different devices, but debugging the application when something does not work as expected is still a difficult task.

\subsubsection{XDSession}

XDSession, developed by Nebeling et al.~\cite{xdsession2015}, is a framework for cross-device application development based on the concept of \emph{cross-device sessions} which is also useful for logging and debugging. The session controller supports management and testing of cross-device sessions with connected or simulated devices at run time. The session inspector enables inspection and analysis of multi-device/multi-user sessions with support for deterministic record/replay of cross-device sessions. A session consists of users, devices, and information.

XDSeession provides a capture and replay mechanism for user interactions and changes to the sessions and provides a basic device emulation mode that allows developers to emulate one device at a time. However, the record and replay mechanism only works if the framework API is used for the manipulations. Also, XDSession supports debugging at a rather high level of abstraction and is thus better suited for finding problems in interactions rather than bugs in the source code.

\subsubsection{Weave}

Weave is a web-based framework for creating cross-device wearable interaction by scripting, developed by Chi et al.~\cite{weave2015}. It provides a set of high-level APIs for developers to easily distribute UI output and combine sensing events and user input across mobile and wearable devices. Devices can be manipulated regarding their capabilities and affordances, rather than low-level specifications. Weave also has an integrated authoring environment for developers to program and test cross-device behaviors. Developers can test their scripts based on a set of simulated or real wearable devices. Weave's APIs capture affordances of wearable devices and provide mechanisms for distributing output and combining sensing events and user input across multiple devices.

Weave allows testing on emulated and real devices and the developer also has access to a log panel, but no further support for debugging is provided. 

\subsubsection{WatchConnect}

In ~\cite{watchconnect2015}, Houben et al. present WatchConnect. WatchConnect is a toolkit for rapidly prototyping cross-device applications and interaction techniques with smartwatches. It provides an extendable hardware platform that emulates a smartwatch, a UI framework that integrates with an existing UI builder and a rich set of input and output events using a range of built-in sensor mappings. WatchConnect is built around a wired prototyping watch, a smart watch emulator composed of a display, a number of sensors and a microprocessor. Applications can thus be tested and debugged directly on the watch prototype. WatchConnect also includes some tools for supporting developers while debugging, but those tools are focused on the machine learning algorithms and the recording of sensor data.

\subsection{Frameworks that would Benefit from Testing Tools}

The following frameworks include no tools for testing and debugging the applications developed with them and would certainly benefit from such tools.

\subsubsection{XD-MVC}

XD-MVC is a framework that combines cross-device capabilities with MVC frameworks. It can be used as a plain JavaScript library or in combination with Polymer. The framework consists of a server-side and a client-side part. For communicating, either a \emph{peer-to-peer} or a \emph{client-server} approach can be used. Developers that use XD-MVC can assign \emph{roles} to devices that are connected to the application. Depending on the role that a device is assigned, different parts of the interface are shown on the device. 

\subsubsection{Connichiwa}

Connichiwa is a framework for developing cross-device web applications developed by Schreiner at al.~\cite{connichiwa2015}. It runs local web applications on one of the devices without requiring an existing network or internet connection. \emph{No remote server} is used, instead a native helper-application runs a web server on-demand. The native application automatically \emph{detects other devices} using Bluetooth Low Energy, which are then connected by sending the IP address of the local web server over Bluetooth. Connichiwa's \emph{JavaScript API} gives easy access to common functions like device detection and connection and it also provides JavaScript events to notify about device detection and connection.

\subsubsection{DireWolf}

DireWolf is a framework for distributed web applications based on widgets. It was developed by Kovachev et al.~\cite{direwolf2013}. Widgets can be shared, reused, mashed up and personalized between applications. Splitting the interface into separate widgets and enabling them to exchange information allows the development of customizable web applications. Direwolf provides a \emph{framework for easy browser-based distribution} of web widgets between multiple devices, facilitates extended \emph{multi-modal real-time interactions} on a federation of personal computing devices and provides \emph{continuous state-preserving widget migration}. DireWolf helps managing a set of devices and handles communication and control of distributed parts of the web application. 

\subsubsection{Panelrama}

Panelrama, developed by Yang et al.~\cite{panelrama2014}, is a web-based framework for the construction of applications using distributed user interfaces. It introduces a new XML element, "panel", which may be placed around groupings of control and it facilitates the \emph{distribution and synchronization of panels} among the connected devices. The developer can specify the state information that should be synchronized across devices as well as the suitability of panels to different types of devices. An \emph{optimization algorithm} distributes panels to devices that maximize their match for the developer's intent.

\subsubsection{Polychrome}

Polychrome is a web application framework for creating web-based collaborative visualizations that can span multiple devices. It was developed by Badam et al.~\cite{polychrome2014}. It supports co-browsing new web applications as well as legacy websites with \emph{no migration costs}, an \emph{API to develop new web applications} that can synchronize the UI state on multiple devices to support synchronous and asynchronous collaboration and \emph{maintenance of state and input events} on a server to handle common issues with distributed applications. Polychrome provides the interaction and display space distribution mechanisms to create new collaborative web visualizations that utilize multiple devices and it provides framework modules to store the user interaction. Combined with the initial state of the website, the interaction logs are useful for synchronizing devices within the collaborative environment, consistency management and interaction replay.

\subsection{Summary}

In the sections above, we have described several different frameworks for developing cross-device applications. Some of them are only useful for specific types of applications, while others are suited for cross-device applications of all types. Despite this abundance of frameworks available, many of those frameworks provide no support for testing and debugging the applications that are developed with them. DireWolf, Polychrome and XD-MVC all include no features for testing and run in an unmodified browser. Thus, they could all benefit from a web-based tool for testing cross-device applications. Panelrama applications also run in an unmodified browser. The developers of Panelrama mention that a tool for simulating device configurations and previewing the distribution was built on request of some developers. No further details about the tool were mentioned, but the fact that developers requested such a tool shows that it is often difficult to imagine what an application would look like in different device configurations. Thus, the developers of Panelrama applications would certainly also benefit from a cross-device testing tool. Connichiwa requires the installation of a helper application, but the application itself still runs in an unmodified browser. Thus, applications developed with Connichiwa could also be tested with a web-based cross-device testing tool.

Some of the frameworks above allow the emulation of devices as well as the connection of real devices. Emulating multiple devices at a time as well as connecting real devices is a crucial feature for testing and debugging cross-device applications, but without any additional tools that help with debugging, finding and fixing bugs in the applications is still a difficult task. The framework that provides the most support for testing so far is probably XDSession, but as described above, its mechanisms are still not enough for successfully debugging cross-device applications at the source code level.