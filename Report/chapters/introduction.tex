\chapter{Introduction}

Despite the abundance of devices nowadays, up until recently there was no easy way of sharing state information and I/O resources between devices for the end user. Although many users have access to multiple devices at the same time, e.g. their smartphone and laptop, those devices are mostly used independently. Santosa et al.~\cite{santosa2013} have observed in a field study that many users already use multiple devices in parallel in their workflows and that better functional coordination is needed. In the last few years, cross-device applications have started to fill this gap by facilitating the use of multiple devices at once and the sharing of data between them. The emergence of new web technologies such as Device APIs\footnote{\url{http://www.w3.org/2009/dap/}} and WebRTC\footnote{\url{http://www.webrtc.org/}} encouraged the development of a new generation of web-based frameworks that facilitate the development of such cross-device applications. Cross-device applications typically run on any device that has access to a modern web browser. 

Despite the large number of frameworks for developing cross-device applications and the identified user needs, there are only few popular cross-device applications. Many of the available cross-device applications are prototypes to showcase the frameworks that they were developed with and most of them are not accessible to the public. In order to release cross-device applications into the wild, they need to be carefully tested and bugs need to be eliminated. However, existing cross-device application development frameworks have little support for this and either provide no facilities for testing applications at all or only very basic facilities that are focused on specific aspects of the applications, such as seeing what the application looks like on specific sets of devices.

In contrast, there are already plenty of practical tools for testing and debugging traditional web applications and many of them can be accessed directly from modern browsers. Google Chrome\footnote{\url{http://www.google.com/chrome/}} in particular, but also some other browsers, provide quite mature tools for debugging JavaScript, HTML and CSS. Unfortunately, those tools are focused on debugging one device at a time, limiting their usefulness for testing cross-device applications where multiple devices are typically involved simultaneously. Today's devices have many different characteristics, mainly in terms of screen size, but also concerning their input capabilities and connectivity. This diversity of devices requires developers to develop websites that are functional and appealing on all devices. This goal can be achieved by following the principles of responsive design. Many tools have emerged that support testing of such websites; some are already built into modern web browsers while others can be accessed through a website or by installing a program on a desktop PC. In those tools, two different approaches to testing responsive websites can be observed: First, different devices can be emulated on a desktop computer. Second, a varied set of actual devices can be used. Google Chrome's Device Mode\footnote{\url{https://developer.chrome.com/devtools/docs/device-mode}} provides extensive support for emulating devices; apart from simply emulating the screen size, it can also emulate touch, varying network conditions, location, and more. Other browsers also provide basic facilities for device emulation. When using multiple devices, the developer has to refresh all devices individually whenever the web application has been modified. However, there exist a number of tools that facilitate this. Some allow the developer to reload all devices at once, e.g. Adobe Edge Inspect CC\footnote{\url{https://www.adobe.com/ch_de/products/edge-inspect.html}}, while others also automatically reload devices when files change, e.g. BrowserSync\footnote{\url{http://www.browsersync.io/}}. Some of those tools even allow developers to simultaneously browse their web application on multiple devices. Apart from those tools, there are also web services for testing websites across multiple devices and platforms. Such web services typically include a screenshot generation service that renders a given website on a large number of devices. An example of such a web service is CrossBrowserTesting\footnote{\url{http://crossbrowsertesting.com/}}.

Those tools are already a good starting point for testing and debugging cross-device applications. However, web applications targeted at one device at a time and cross-device applications have some fundamental differences that are not accounted for by those tools. In cross-device applications, multiple devices are typically used simultaneously and in a coordinated manner. Also, different devices do not necessarily show the same thing in cross-device applications, which limits the use of mirroring interactions from one device to all other devices. Furthermore, most of the tools for emulating devices focus on emulating one device at a time, which requires the developer to open multiple browser windows, possibly with different user profiles or in incognito mode to prevent the sharing of local resources. Finally, all those tools focus either on emulating devices or on using real devices, but with cross-device applications, it might be desirable to combine both approaches. Due to those differences, testing and debugging cross-device applications is still a challenging task. In the following section, we will describe how our project contributes to conquering the challenges in cross-device application testing and debugging.

\section{Contributions}

Our project aims to facilitate the testing and debugging of cross-device applications by providing multiple tools that assist the developer. As a first step towards achieving our goal, we analyzed existing tools for debugging web applications. This includes tools built directly into browsers as well as external tools. Due to the similarities between cross-device applications and responsive web applications, in particular the fact that both are supposed to work on devices with many different characteristics, tools for testing responsive websites are of particular interest. During our analysis, we gathered the limitations of those tools and possible remedies to those limitations. Furthermore, we also investigated some frameworks for developing cross-device applications. The analysis of those frameworks showed that even though many frameworks could benefit from tools assisting them with testing the applications developed with them, few such tools are provided by the frameworks. The limited number of available tools makes exhaustive testing of a cross-device application a difficult task. Finally, we gathered some requirements for more suitable tools for cross-device application testing based on the limitations of existing tools.

Based on those requirements, we designed and implemented a new set of tools, called XDTools, for testing and debugging cross-device applications. XDTools allows testing applications both on real devices and emulated devices and provides a number of different features: Some features have been directly adopted from existing tools while others are based on the features of some existing tools but have been extended to suit the needs of cross-device application testing. Some features that are only useful in a cross-device environment are completely new and not based on any available features.

During the development of XDTools, we also implemented two sample cross-device applications using XD-MVC\footnote{\url{https://github.com/mhusm/XD-MVC}}, a cross-device application development framework. While developing those applications, we used XDTools for testing and debugging our applications. Those applications helped us in multiple ways while developing XDTools: First, actually using XDTools for developing cross-device applications provided some new ideas for crucial features that were still missing from XDTools. Second, developing sample applications helped us improve existing features and also revealed some bugs that could be fixed. Finally, we also learned something about the usability of XDTools.

Eventually, we conducted a user study to evaluate the usefulness and suitability of XDTools, compared to traditional methods of testing web applications. During this study, participants got the opportunity to use XDTools for implementing a new feature in a cross-device application and for finding and fixing a bug. The study showed that XDTools is indeed considered useful for testing cross-device applications. Furthermore, some participants also provided some new ideas for features and improvements for our existing features as well as the layout of XDTools.

In summary, our project makes the following contributions:
\begin{itemize}
	\item Analysis of the limitations of existing tools for web application testing in general as well as responsive design testing.
	\item Development of XDTools, a set of tools for testing and debugging cross-device applications.
	\item Development of two sample cross-device applications.
	\item User study for evaluating XDTools.
\end{itemize}

\section{Structure of this Document}

We conclude this chapter by giving an overview of the structure of this document:

In Chapter 2, we present some background information as well as related work to our project. In particular, we describe existing tools for testing web applications as well as cross-device application development frameworks.

In Chapter 3, we describe the requirements for XDTools that we gathered during our analysis.

In Chapter 4, we present XDTools.

In Chapter 5, we describe the architecture and implementation of XDTools in detail.

In Chapter 6, we describe the sample applications that we developed and the insights we gained from them.

In Chapter 7, we describe the user study and present its results.

In Chapter 8, we conclude the work by showing what was achieved and what problems remain, and address possible future work that might be based on XDTools.