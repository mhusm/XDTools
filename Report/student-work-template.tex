\documentclass[11pt,a4paper]{globis-book}

\usepackage{graphicx}
\usepackage[helvetica]{quotchap}
\usepackage{times}
\usepackage{ethfont}
\usepackage[british]{babel}
\usepackage{longtable}
\usepackage{fancyhdr}
\usepackage{shadethm}
\usepackage{makeidx}
\usepackage[a4paper,portrait,twoside,inner=3.25cm,outer=3.5cm,top=3.5cm,bottom=4.0cm]{geometry}
\usepackage{globis}
\usepackage{float}
\usepackage{listings}
\usepackage{pdfpages}
\usepackage{subcaption}
\usepackage{caption}
\usepackage{hyperref}

\pdfoptionpdfminorversion=6
\pdfsuppresswarningpagegroup=1

\renewcommand{\sectfont}{\sffamily\bfseries\Huge}

\setlength{\shadedtextwidth}{\textwidth}
\setlength{\shadeleftshift}{3mm}
\setlength{\shaderightshift}{3mm}
\addtolength{\shadedtextwidth}{-\shadeleftshift}
\addtolength{\shadedtextwidth}{-\shaderightshift}
\setlength{\parindent}{0pt}
\setlength{\parskip}{5pt}

\lstset{ %
  aboveskip=\bigskipamount,			     % top margin
	belowskip=\bigskipamount,			     % bottom margin
  basicstyle=\ttfamily,       	   	 % the size of the fonts that are used for the code
  numbers=left,                      % where to put the line-numbers
  numberstyle=\tiny\color{gray},     % the style that is used for the line-numbers
  stepnumber=1,                      % the step between two line-numbers. If it's 1, each line will be numbered
  numbersep=5pt,                     % how far the line-numbers are from the code
  backgroundcolor=\color{white},     % choose the background color. You must add \usepackage{color}
  showspaces=false,                  % show spaces adding particular underscores
  showstringspaces=false,            % underline spaces within strings
  showtabs=false,                    % show tabs within strings adding particular underscores
  frame=single,                      % adds a frame around the code
  rulecolor=\color{black},           % if not set, the frame-color may be changed on line-breaks within not-black text (e.g. comments)
  tabsize=4,                         % sets default tabsize to 2 spaces
  captionpos=t,                      % sets the caption-position to bottom
  belowcaptionskip=\smallskipamount, % margin below caption
  breaklines=true,                   % sets automatic line breaking
  breakatwhitespace=false,           % sets if automatic breaks should only happen at whitespace
  title=\lstname,                    % show the filename of files included with \lstinputlisting; also try caption instead of title
  keywordstyle=\color{blue},         % keyword style
  commentstyle=\color{comments},     % comment style
  stringstyle=\color{strings},       % string literal style
  identifierstyle=\color{black},     % identifier style
  escapeinside={(*@}{@*)}            % if you want to add LaTeX within your code
}

\sloppy

\pagestyle{fancy}
\fancyhf{}
\fancyhead[LE,RO]{\sffamily\bfseries\small\thepage}
\fancyhead[LO]{\sffamily\bfseries\small\leftmark}
\fancyhead[RE]{\sffamily\bfseries\small\rightmark}
\renewcommand{\headrulewidth}{0.1pt}
\renewcommand{\footrulewidth}{0pt}

\fancypagestyle{plain}{
   \fancyhf{}
   \fancyfoot[C]{\sffamily\bfseries\small\thepage}
   \renewcommand{\headrulewidth}{0pt}
   \renewcommand{\footrulewidth}{0pt}
}

% Please adapt the following fields if necessary!
\hypersetup{
    pdftitle = XDTools: Testing and Debugging Cross-Device Web Applications
    pdfauthor = Nina Heyder,
    pdfsubject = Master Thesis,
		hidelinks,
    plainpages = false,
    bookmarksnumbered = true
} 

\raggedbottom

\title{XDTools: Testing and Debugging Cross-Device Applications}
\category{Master Thesis}
\author{Nina Heyder}
\email{$<$heydern@student.ethz.ch$>$}
\professor{Prof. Dr. Moira C. Norrie}
\assistant{Maria Husmann}
\group{Global Information Systems Group}
\institute{Institute of Information Systems}
\department{Department of Computer Science}
\school{ETH Zurich}
\version{}
\date{\today}
\copyrightyear{2015}

\makeindex

\begin{document}

\frontmatter
\maketitlepage
\cleardoublepage
\pdfbookmark{Contents}{toc}

\chapter*{Abstract}

Nowadays, many people have access to multiple devices simultaneously. However, those devices are mostly used independently. Cross-device applications aim to enable better collaboration between devices by distributing interfaces among devices and synchronizing data between them. Recently, many frameworks for developing such applications have been created. Despite this abundance of frameworks, cross-device applications are rarely used in the wild. One reason for the weak proliferation of cross-device applications could be the limited support for testing and debugging them. Without proper testing, releasing a high-quality product is a risky undertaking. Our analysis has shown that existing cross-device frameworks provide little to no support for testing and debugging. Also, debugging tools for testing traditional and responsive web applications are typically not well-suited for debugging multiple devices simultaneously. In this thesis, we have developed XDTools, a set of tools that support the testing and debugging of web-based cross-device applications. XDTools extends traditional debugging tools with support for multiple devices, like the JavaScript console available in browsers, and also introduces some new tools, for example connection management. In order to inform the design and as a preliminary evaluation of XDTools, we have developed two sample applications that were tested with XDTools during development. We have also conducted a user study where participants used XDTools to complete tasks in a cross-device environment. The study showed promising results and we have received enthusiastic feedback. Most of the participants appreciated the tools provided by XDTools and found that they were well-suited for cross-device application testing. Thus, XDTools improves the debugging process of cross-device applications and serves as a basis for further research in cross-device application testing.

\tableofcontents

\mainmatter

\chapter{Introduction}

Despite the abundance of devices nowadays, up until recently there was no easy way of sharing state information and I/O resources between devices for the end user. Although many users have access to multiple devices at the same time, e.g. their smartphone and laptop, those devices are rarely used in collaboration. Santosa et al.~\cite{santosa2013} have observed in a field study that many users already use multiple devices in parallel in their workflows and that better functional coordination is needed. In the last few years, cross-device applications have started to fill this gap by facilitating the use of multiple devices in collaboration. Cross-device applications typically run on any device that has access to a modern web browser. The emergence of new web technologies such as Device APIs\footnote{\url{http://www.w3.org/2009/dap/}} and WebRTC\footnote{\url{http://www.webrtc.org/}} encouraged the development of a new generation of web-based frameworks that facilitate the development of such cross-device applications.

Despite the large number of frameworks for developing cross-device applications and the identified user needs, there are only few popular cross-device applications. Many of the available cross-device applications are prototypes to showcase the frameworks that they were developed with and most of them are not accessible to the public. In order to release cross-device applications into the wild, they need to be carefully tested and bugs need to be eliminated. However, existing cross-device frameworks have little support for this and either provide no tools for testing applications at all or only very basic tools that are focused on specific aspects, such as seeing what the application looks like on different sets of devices.

In contrast, there are already plenty of practical tools for testing and debugging traditional web applications and many of them can be accessed directly from modern browsers. Google Chrome\footnote{\url{http://www.google.com/chrome/}} in particular, but also other browsers, provide quite mature tools for debugging JavaScript, HTML and CSS. Unfortunately, those tools are focused on debugging one device at a time, limiting their usefulness for testing cross-device applications where multiple devices are typically involved simultaneously. Today's devices have many different characteristics, mainly in terms of screen size, but also concerning their input capabilities and connectivity. This diversity of devices requires developers to develop websites that are functional and appealing on all devices. This goal can be achieved by following the principles of responsive design. Many tools have emerged that support testing such websites; some are already built into modern web browsers, others can be accessed as a web service or by installing a program on a desktop PC. In those tools, two different methods for testing responsive websites can be observed: First, different devices can be emulated on a desktop PC. Second, a varied set of actual devices can be used. Google Chrome's Device Mode\footnote{\url{https://developer.chrome.com/devtools/docs/device-mode}} provides extensive support for emulating devices; apart from emulating the screen size, it can also emulate touch, varying network conditions, location, and more. Other browsers also provide basic facilities for device emulation. When using multiple devices, the developer has to refresh all devices individually whenever the web application has been modified. However, there exist a number of tools that facilitate this. Some allow the developer to reload all devices at once, e.g. Adobe Edge Inspect CC\footnote{\url{https://www.adobe.com/ch_de/products/edge-inspect.html}}, while others also automatically reload devices when files change, e.g. BrowserSync\footnote{\url{http://www.browsersync.io/}}. Some of those tools even allow developers to simultaneously browse their web application on multiple devices. Apart from those tools, there are also web services for testing websites across multiple devices and platforms. Such web services typically include a screenshot generation service that renders a given website on a large number of devices. An example of such a web service is CrossBrowserTesting\footnote{\url{http://crossbrowsertesting.com/}}.

Those tools are already a good starting point for testing and debugging cross-device applications. However, web applications targeted at one device at a time and cross-device applications have some fundamental differences that are not accounted for by those tools. In cross-device applications, multiple devices are typically used simultaneously and in a coordinated manner. Also, different devices do not necessarily show the same thing in cross-device applications, which limits the use of mirroring interactions from one device to all other devices. Furthermore, most of the tools for emulating devices focus on emulating one device at a time, requiring the developer to open multiple browser windows, possibly with different user profiles or in incognito mode to prevent the sharing of local resources. Finally, all those tools focus either on emulating devices or on using real devices, but with cross-device applications, it might be desirable to combine both approaches. Due to these differences, testing and debugging cross-device applications is still a challenging task. In the following section, we will describe how our project contributes to conquering the challenges in cross-device application testing and debugging.

\section{Contributions}

Our project aims to facilitate the testing and debugging of cross-device applications by providing multiple tools that assist the developer. As a first step towards achieving our goal, we have analyzed existing tools for debugging web applications. This includes tools built directly into browsers as well as external tools. Due to the similarities between cross-device applications and responsive web applications, e.g. the fact that both are supposed to work on devices with many different characteristics, tools for testing responsive websites are of particular interest. During our analysis, we have gathered the limitations of those tools and possible remedies for these limitations. Furthermore, we have also investigated some frameworks for developing cross-device applications. The analysis of those frameworks has shown that even though many frameworks could benefit from cross-device testing tools, few such tools are provided by the frameworks themselves. The limited number of available tools makes exhaustive testing of a cross-device application a difficult task. Finally, we have gathered some requirements for more suitable tools for cross-device application testing based on the limitations of existing tools.

Based on these requirements, we have designed and implemented a new set of tools, called XDTools, for testing and debugging cross-device applications. XDTools allows testing both on real devices and emulated devices and provides a number of different features: Some features have been directly adopted from existing tools while others have been extended to suit the needs of cross-device application testing. Some features that are only useful in a cross-device environment are completely new and not based on any existing tools.

During the development of XDTools, we have implemented two sample cross-device applications using XD-MVC\footnote{\url{https://github.com/mhusm/XD-MVC}}, a cross-device application development framework. While developing these applications, we have used XDTools for testing and debugging them. These applications have helped us in multiple ways while developing XDTools: First, actually using XDTools for developing cross-device applications has provided some new ideas for crucial features that were still missing from XDTools. Second, it has helped us improve existing features and also revealed some bugs that could be fixed. Finally, we have also learned something about the usability of XDTools.

Eventually, we have conducted a user study to evaluate the usefulness and suitability of XDTools, compared to traditional methods of testing web applications. During this study, participants got the opportunity to use XDTools for implementing a new feature in a cross-device application and for finding and fixing a bug. The study has shown that XDTools is indeed considered useful for testing cross-device applications. Furthermore, some participants have also provided some new ideas for features and improvements for our existing features as well as the layout of XDTools.

In summary, our project makes the following contributions:
\begin{itemize}
	\item Analysis of the limitations of existing tools for web application testing in general as well as responsive design testing.
	\item Development of XDTools, a set of tools for testing and debugging cross-device applications.
	\item Development of two sample cross-device applications.
	\item User study for evaluating XDTools.
\end{itemize}

\section{Structure of This Document}

We conclude this chapter by giving an overview of the structure of this document:

In Chapter 2, we present some background information as well as related work to our project. In particular, we describe existing tools for testing web applications as well as cross-device application development frameworks.

In Chapter 3, we describe the requirements for XDTools that we gathered during our analysis.

In Chapter 4, we present XDTools.

In Chapter 5, we describe the architecture and implementation of XDTools in detail.

In Chapter 6, we describe the sample applications that we developed and the insights we gained from them.

In Chapter 7, we describe the user study and present its results.

In Chapter 8, we conclude the work by showing what was achieved and what problems remain, and address possible future work that might be based on XDTools.
\chapter{Background and Related Work}

In the following sections, we will present some background information and related work for our project. We will start by analyzing existing tools for debugging traditional web applications, including tools that are integrated directly into browsers. We will also discuss some tools aimed at testing responsive web applications. Finally, we will present a few cross-device application development frameworks that either provide some cross-device testing facilities themselves or that could benefit from such facilities.

\section{Web Application Testing}

In the following subsections, we will present a few tools supporting the testing and debugging of traditional web applications. Those tools are also relevant for cross-device applications, as most cross-device applications are web applications.

\subsection{Browser-Integrated Debugging Tools}

Over the last few years, browsers have added more and more tools for debugging web applications. Chrome probably provides the most advanced tools, but all browsers include at least some tools for debugging web applications. We will now present a few of those tools provided by browsers themselves.

\subsubsection{Chrome DevTools}

Google Chrome already provides a wide range of features for testing and debugging web applications. First of all, it lets the developer inspect the DOM tree and allows inspection and on-the-fly editing of DOM elements. Furthermore, new CSS rules or properties can be added or existing ones can be modified. Figure~\ref{fig:chrome_devtools} shows a screenshot of the HTML and CSS inspector in Chrome DevTools. Another useful feature is the JavaScript Console. It has two main purposes: First, it can be used to log diagnostic information in the development process. Second, it is a shell prompt which can be used to interact with the document and DevTools. Chrome DevTools can also be used to debug JavaScript. In the DevTools, all scripts that are part of the inspected page can be seen. Breakpoints can be set in the scripts and standard controls to pause, resume and step through code are provided. All features mentioned before are also available for debugging remote devices. Remote debugging can be used by connecting a device to the desktop PC with a cable.

\begin{figure}[H]
  \centering
    \includegraphics[width=1.0\textwidth]{images/relatedwork/chrome_devtools_3.png}
	\caption[Screenshot: Chrome DevTools]{HTML/CSS inspector of Chrome DevTools}
	\label{fig:chrome_devtools}
\end{figure}

Chrome DevTools provides many more features useful for debugging, but explaining them all would exceed the scope of this document. While all those features are very useful for testing and debugging web applications, the main disadvantage is that they can only be used to debug one device and one website at a time.

\subsubsection{Firefox Developer Tools}

Firefox\footnote{\url{https://developer.mozilla.org/en-US/docs/Tools}} provides developer tools similar to Chrome DevTools. It also has a page inspector that allows developer to inspect the DOM tree and modify the CSS. It also features a console, a JavaScript debugger and remote debugging. Like Google Chrome's DevTools, Firefox's developer tools provides more features that will not be explained here. Furthermore, Firefox gives developers access to a feature called Developer Toolbar. The Developer Toolbar is a command-line tool that can be used to access a number of developer tools from within Firefox. Overall, Chrome DevTools and Firefox Developer Tools are very similar and thus also have the same limitations. Just like Google Chrome, only one device can be debugged at a time in Firefox.

\subsubsection{Summary}

Internet Explorer, Microsoft Edge, Opera, Safari and other browsers all provide some kind of developer tools. However, they do not provide any features that we did not already mention before, thus we will not provide any detailed description of the developer tools available in those browsers. All those browsers share the same limitation: Their tools can only be used to debug one device at a time. For cross-device applications however, it would be desirable to debug multiple devices at a time without having to navigate between windows all the time.

\subsection{Record and Replay}

Record and replay has already successfully been used for debugging web applications. It allows developers to quickly reproduce bugs that are otherwise difficult to reproduce. We will investigate a few tools that support record and replay in web applications.

\subsubsection{Timelapse}

In~\cite{timelapse2013}, Burg et al. describe their record and replay tool Timelapse. Timelapse is a tool for recording, reproducing, and debugging interactive behaviors in web applications. Timelapse is built on Dolos, a record/replay infrastructure that ensures deterministic execution by capturing and reusing user inputs, network responses, and other non-deterministic inputs. Developers can use Timelapse to browser, visualize, and seek within recorded program executions while simultaneously using familiar debugging tools such as breakpoints and logging. Figure~\ref{fig:timelapse} shows Timelapse while being used with the debugging tools. The developers of Timelapse conducted a user study with 14 web developers which showed no significant effect on task times, task success, or time spent reproducing behaviors when developers had access to Timelapse. Expert developers seemed to better integrate Timelapse into their workflow, using Timelapse to accelerate familiar tasks rather than redesigning their workflow. However, Timelapse distracted less-skilled developers that were led astray by unverified assumptions.

\begin{figure}[H]
  \centering
    \includegraphics[width=1.0\textwidth]{images/relatedwork/timelapse.png}
	\caption[Screenshot: Timelapse]{Screenshot of Timelapse}
	\label{fig:timelapse}
\end{figure}

\subsubsection{Mugshot}

Mugshot, developed by Mickens et al.~\cite{mugshot2010}, is a record/replay system that captures all events in a JavaScript program, allowing developers to deterministically replay past executions of web applications. The goal of Mugshot is to provide low-overhead, "always-on" capture and replay for web-deployed JavaScript programs. Mugshot logs explicit user interactions like mouse clicks as well as background activities such as random number generation and the firing of timer callbacks. The client-side log is sent to the developer in response to a trigger like an unexpected exception being caught. The developer can then use Mugshot's replay mode to recreate the original JavaScript execution on his unmodified browser. 

\subsubsection{WaRR}

WaRR~\cite{warr2011} is a high-fidelity, "always-on" tool that records and replays the interactions between users and web applications. The WaRR recorder is embedded directly into the web browser and the WaRR Replayer uses an enhanced, developer-specific web browser that enables more realistic simulation of user interactions based on the recorded traces. Thus, the recording functionality is an integral part of the browser. Andrica et al. developed two tools on top of WaRR:
\begin{itemize}
	\item WebErr allows testing of web applications against human errors, i.e. navigation errors and timing errors.
	\item AUsER automatically generates user experience reports: If a user experiences a bug while using a web application, they press a button and the developers of the application receive the sequence of actions that led to the bug.
\end{itemize}
Using WebErr, the developers of WaRR were able to find a bug in Google Sites\footnote{\url{https://www.google.com/work/apps/business/products/sites/}}.

\subsubsection{Rumadai}

Yildiz et al.~\cite{rumadai2012} developed Rumadai, a Visual Studio plug-in that helps developers test web applications by recording and replaying client-side events. Rumadai injects JavaScript code into web pages to be deployed at servers. The injected code records user events as well as client-side dynamic content requests and their responses. The recorded events are sent to a database and can be queried by the developers of the web page. The recorded events can then be replayed in a browser using Rumadai seamlessly from Visual Studio.

\subsubsection{FireCrystal}

FireCrystal~\cite{firecrystal2009} is a Firefox extension that allows developers to extract the implementation details of interactive behaviors from other websites. Developers can tell FireCrystal to start recording and demonstrate the interactive behavior they want to extract. FireCrystal records the interaction, keeping track of DOM changes, JavaScript executions and user input events. The developer can then replay the interactions and FireCrystal displays the HTML, CSS and JavaScript code that affected a particular element at any specific time. FireCrystal also provides an execution timeline that developers can scrub back and forth.

\subsubsection{Summary}

The tools described above all employ record and replay in some way for improving web applications. FireCrystal focuses more on extracting interactive behaviors from other websites, but it also provides basic record and replay mechanisms. The other tools all focus on debugging web applications. WaRR and Mugshot provide recording mechanisms directly to users of web applications, providing them with means of submitting bug reports to the developers of the applications. Some of those tools are designed for only replaying event sequences on the devices they were recorded, while others allow replaying event sequences on different devices, e.g. the sequences can be replayed on the developer's machine after a user has recorded the bug. However, all of those tools have one limitation in common: They focus on replaying an event sequence on one device at a time. 

Record and replay mechanisms have already been shown to help with debugging in traditional web applications, thus they could certainly also be useful for debugging cross-device applications. In particular, we believe that replaying event sequences on multiple devices simultaneously could help simulating multiple users using a cross-device application. Developers usually work alone when fixing a bug, or in very small groups, which makes reproducing bugs that require multiple devices to interact simultaneously a very difficult task. If multiple devices should replay event sequences simultaneously, mechanisms for accurately timing event execution are also required. Most of the tools described above do not include such mechanisms, or they include some mechanisms that are however not sufficient for configuring timing in a cross-device replaying scenario.

\section{Responsive Web Application Testing}

There are a number of tools for testing responsive websites. Some of them can be accessed on the web as a service, while others are desktop programs that have to be installed. Some focus more on testing on real devices and others focus on emulating devices. In the following subsections we will describe some of those tools. Responsive web applications have a few similarities with cross-device applications: Both types of applications are supposed to work on a number of different devices with different screen sizes, input capabilities and more. Thus, tools for testing responsive web applications are to some extent also suited for testing cross-device applications.

\subsection{Web Services}

The following two tools can be accessed through a browser and do not have to be installed.

\subsubsection{BrowserStack}

BrowserStack\footnote{\url{https://www.browserstack.com/}} allows developers to select browsers and devices and then generate screenshots. It is also possible to live test one device at a time. Furthermore, there are developer tools for remote devices and Selenium cloud testing is possible. Advantages of BrowserStack are that real iOS devices are used for screenshots and interactions can be tested automatically using Selenium. A major disadvantage is that only emulated Android devices are available, which limits the usefulness of the tool. Furthermore, only one device can be live tested at a time which is a huge disadvantage for cross-device application testing. 

\subsubsection{CrossBrowserTesting}

With CrossBrowserTesting, developers can select a number of devices as well as the operating system, browser and resolution and generate screenshots. The layout differences between different devices can then automatically be analyzed. Furthermore, websites can also be live tested and Selenium automated testing is available as well. The main advantage of CrossBrowserTesting is that all screenshots are generated on real devices and a very wide variety of devices is available. Also, interactions can be tested. The usefulness of the tool is again limited by the fact that live testing is only possible on one device at once. Additionally, while detecting layout differences is a useful feature in general, it is not yet very mature and some layout differences that are detected seem rather trivial (the body element of a larger device is larger), while other differences are not noticed at all. Figure~\ref{fig:crossbrowsertesting} shows CrossBrowserTesting while taking screenshots on multiple devices.

\begin{figure}[H]
  \centering
    \includegraphics[width=1.0\textwidth]{images/relatedwork/cross_browser_testing.png}
	\caption[Screenshot: CrossBrowserTesting]{Taking screenshots on multiple devices simultaneously}
	\label{fig:crossbrowsertesting}
\end{figure}

\subsection{Testing with Actual Devices}

Using the following tools, developers can connect and debug real devices. 

\subsubsection{Remote Preview}

Remote Preview\footnote{\url{https://github.com/viljamis/Remote-Preview}} allows synchronizing URLs across multiple devices. This allows fast previewing of a website on multiple devices. In cross-device scenarios, it may provide especially useful for quick connecting of devices in applications where devices are connected simply by copying the same URL to all devices. However, the fact that this tool only provides one feature, namely URL synchronizing, limits its usefulness.

\subsubsection{BrowserSync}

BrowserSync (see Figure~\ref{fig:browsersync}) provides a large number of features for testing websites. It allows remote debugging of HTML and CSS, can add CSS outlines or box shadows to all elements and add a CSS grid overlay. It can also load a URL on all devices or refresh all devices as well as automatically refresh devices when files are changed. Furthermore, it allows synchronizing interaction between devices, i.e. clicks, scrolls, form submits, form inputs and form toggles. It also provides network throttling. The advantages of BrowserSync are that it provides a wide range of features, including synchronizing interactions, which is a feature that distinguishes it from other tools. However, for cross-device application testing, synchronizing interactions among all devices is of limited usefulness, as different devices have different roles and thus also different responsibilities.

\begin{figure}[H]
  \centering
    \includegraphics[width=1.0\textwidth]{images/relatedwork/browser_sync_3.png}
	\caption[Screenshot: BrowserSync]{Screenshot of BrowserSync}
	\label{fig:browsersync}
\end{figure}

\subsubsection{Adobe Edge Inspect CC}

Adobe Edge Inspect CC allows developers to take screenshots on all connected devices simultaneously. The screenshots are then automatically transferred to a folder on the desktop PC. It can also refresh all devices simultaneously. Furthermore, the URL that is opened on the desktop PC is loaded on all other connected devices. It also allows remote HTML and CSS debugging using weinre\footnote{\url{https://people.apache.org/~pmuellr/weinre-docs/latest/}}. The main advantage that distinguishes Adobe Edge Inspect CC from other tools is that is simply synchronizes the URL that the developer is currently looking at on the desktop PC, thus it works even if the developer switches tabs or browser windows. However, the fact that the URL cannot be changed from devices other than the desktop PC could be a disadvantage in some scenarios. Also, the installation process is rather extensive: A program needs to be installed on the desktop PC as well as a Chrome extension and an app needs to be installed on all mobile devices that the developer wants to connect. Regarding cross-device applications, again, refreshing all devices at once can be useful as well as synchronizing URLs in some cases, but other than that, it does not provide any features that help with cross-device application testing.

\subsubsection{Ghostlab}

Ghostlab\footnote{\url{http://www.vanamco.com/ghostlab/}} is one of the more mature tools for website testing and provides features similar to Browser Sync. It can also load a URL on all devices or refresh all devices at once and refresh automatically when files are changed. It provides means for synchronized browsing as well as synchronized HTML and CSS inspection on multiple devices. Furthermore, it can automatically fill out forms and provides remote Javascript debugging. Synchronization can be turned on and off on a per-device basis, which makes the tool more useful than tools that simply synchronize all devices. However, the usefulness is still limited because constantly changing the devices that should be synchronized is time-consuming and error-prone. Also, interactions on cross-device applications typically do not happen in a synchronous fashion.

\subsection{Device Emulation}

The following tools allow developers to emulate devices. 

\subsubsection{Chrome Device Mode}

Google Chrome provides extensive support for emulating devices with its Device Mode (see Figure~\ref{fig:device_mode}). If the developer opens the DevTools, they can switch to the Device Mode and emulate any device. The developer can select an arbitrary device from a large list of predefined devices or create a custom device. They can also enable network throttling, touch emulation and location emulation. The large number of different aspects that are emulated in Device Mode make it very well suited for testing responsive web applications. While most other device emulation tools are limited to emulating resolution and maybe touch, Chrome Device Mode allows to emulate many other aspects as well.

\begin{figure}[H]
  \centering
    \includegraphics[width=0.8\textwidth]{images/relatedwork/device_mode_2.png}
	\caption[Screenshot: Chrome Device Mode]{Screenshot of Chrome Device Mode}
	\label{fig:device_mode}
\end{figure}

\subsubsection{Firefox's Responsive Design View}

Firefox's Responsive Design View also allows developers to emulate devices, similar to Chrome's Device Mode, but it provides fewer features. Instead of focusing on existing devices, it provides a short list of common resolutions that can be selected by the developer. It also allows to emulate a custom resolution. Additionally, the orientation of the devices can be switched, touch can be emulated, and screenshots of the displayed web page can be taken.

\subsection{Summary}

While the tools described above already provide a wide variety of features that are immensely useful for responsive web application testing as well as web application testing in general, the distinguishing characteristics of cross-device applications lead to a limited usefulness of those tools. First of all, there are two different approaches to live testing in the tools described above. The first is to test on real devices, but through a web service. In those tools, live testing is possible on one device at a time, but in cross-device scenarios, multiple devices are typically involved. The second approach is to let the developer connect their own devices and synchronize interactions among all or some devices. This is also problematic for testing cross-device applications because not all devices perform the same interactions and those that do, do not necessarily perform them at exactly the same time. Two features that some of the tools mentioned above provide and that would definitely also be useful for cross-device applications are refreshing all devices at once and loading a URL on all devices. Remote HTML, CSS and JavaScript debugging are also desirable in a cross-device scenario, but this is already covered by Google Chrome anyways. Something similar to Selenium testing would clearly also be useful in cross-device scenarios. However, depending on the device, different tests would be needed and it should be possible to run those tests in parallel.

\section{Cross-Device Application Development Frameworks}

In the following subsections, we will describe some frameworks that facilitate the development of cross-device applications. Some of those frameworks already provide some mechanisms for testing the applications developed with them, but other include none and could benefit a lot from tools that would allows easy testing of cross-device applications.

\subsection{Frameworks that have some Testing Tools}

The following frameworks already include some mechanisms for testing and debugging applications, but most of those mechanisms are limited to specific aspects of the applications and require the applications to be developed with the framework.

\subsubsection{XDStudio}

In ~\cite{xdstudio2014}, Nebeling et al. present their web-based GUI builder, XDStudio (see Figure~\ref{fig:xdstudio}). XDStudio is designed to support interactive development of cross-device web interfaces. It has two complementary authoring modes: \emph{Simulated authoring} allows designing for a multi-device environment on a single device by simulating other target devices. \emph{On-device authoring} allows the design process itself to be distributed over multiple devices. The design process is still coordinated by a main device, but directly involves target devices. The user can switch between two different modes: In \emph{use mode}, the user can interact normally with the interface loaded into the editor. In \emph{design mode}, the user can manipulate the interface directly. XDStudio also allows the specification of \emph{distribution profiles} in terms of involved devices, users and target user interfaces.

\begin{figure}[H]
  \centering
    \includegraphics[width=0.8\textwidth]{images/relatedwork/xdstudio.png}
	\caption[Screenshot: XDStudio]{Screenshot of XDStudio}
	\label{fig:xdstudio}
\end{figure}

Although XDStudio includes mechanisms for inspecting interactive designs on emulated devices and on connected real devices, there is no specific support for debugging such as access to the console. Thus, XDStudio is well suited for seeing what an application looks and feels like on different devices, but debugging the application when something does not work as expected is still a difficult task.

\subsubsection{XDSession}

XDSession, developed by Nebeling et al.~\cite{xdsession2015}, is a framework for cross-device application development based on the concept of \emph{cross-device sessions} which is also useful for logging and debugging. The session controller supports management and testing of cross-device sessions with connected or simulated devices at run time. The session inspector enables inspection and analysis of multi-device/multi-user sessions with support for deterministic record/replay of cross-device sessions. A session consists of users, devices, and information.

XDSeession provides a capture and replay mechanism for user interactions and changes to the sessions and provides a basic device emulation mode that allows developers to emulate one device at a time. However, the record and replay mechanism only works if the framework API is used for the manipulations. Also, XDSession supports debugging at a rather high level of abstraction and is thus better suited for finding problems in interactions rather than bugs in the source code.

\subsubsection{Weave}

Weave is a web-based framework for creating cross-device wearable interaction by scripting, developed by Chi et al.~\cite{weave2015}. It provides a set of high-level APIs for developers to easily distribute UI output and combine sensing events and user input across mobile and wearable devices. Devices can be manipulated regarding their capabilities and affordances, rather than low-level specifications. Weave also has an integrated authoring environment for developers to program and test cross-device behaviors. Developers can test their scripts based on a set of simulated or real wearable devices. Weave's APIs capture affordances of wearable devices and provide mechanisms for distributing output and combining sensing events and user input across multiple devices.

Weave allows testing on emulated and real devices and the developer also has access to a log panel, but no further support for debugging is provided. 

\subsubsection{WatchConnect}

In ~\cite{watchconnect2015}, Houben et al. present WatchConnect. WatchConnect is a toolkit for rapidly prototyping cross-device applications and interaction techniques with smartwatches. It provides an extendable hardware platform that emulates a smartwatch, a UI framework that integrates with an existing UI builder and a rich set of input and output events using a range of built-in sensor mappings. WatchConnect is built around a wired prototyping watch, a smart watch emulator composed of a display, a number of sensors and a microprocessor. Applications can thus be tested and debugged directly on the watch prototype. WatchConnect also includes some tools for supporting developers while debugging, but those tools are focused on the machine learning algorithms and the recording of sensor data.

\subsection{Frameworks that would Benefit from Testing Tools}

The following frameworks include no tools for testing and debugging the applications developed with them and would certainly benefit from such tools.

\subsubsection{XD-MVC}

XD-MVC is a framework that combines cross-device capabilities with MVC frameworks. It can be used as a plain JavaScript library or in combination with Polymer. The framework consists of a server-side and a client-side part. For communicating, either a \emph{peer-to-peer} or a \emph{client-server} approach can be used. Developers that use XD-MVC can assign \emph{roles} to devices that are connected to the application. Depending on the role that a device is assigned, different parts of the interface are shown on the device. 

\subsubsection{Connichiwa}

Connichiwa is a framework for developing cross-device web applications developed by Schreiner at al.~\cite{connichiwa2015}. It runs local web applications on one of the devices without requiring an existing network or internet connection. \emph{No remote server} is used, instead a native helper-application runs a web server on-demand. The native application automatically \emph{detects other devices} using Bluetooth Low Energy, which are then connected by sending the IP address of the local web server over Bluetooth. Connichiwa's \emph{JavaScript API} gives easy access to common functions like device detection and connection and it also provides JavaScript events to notify about device detection and connection.

\subsubsection{DireWolf}

DireWolf is a framework for distributed web applications based on widgets. It was developed by Kovachev et al.~\cite{direwolf2013}. Widgets can be shared, reused, mashed up and personalized between applications. Splitting the interface into separate widgets and enabling them to exchange information allows the development of customizable web applications. Direwolf provides a \emph{framework for easy browser-based distribution} of web widgets between multiple devices, facilitates extended \emph{multi-modal real-time interactions} on a federation of personal computing devices and provides \emph{continuous state-preserving widget migration}. DireWolf helps managing a set of devices and handles communication and control of distributed parts of the web application. 

\subsubsection{Panelrama}

Panelrama, developed by Yang et al.~\cite{panelrama2014}, is a web-based framework for the construction of applications using distributed user interfaces. It introduces a new XML element, "panel", which may be placed around groupings of control and it facilitates the \emph{distribution and synchronization of panels} among the connected devices. The developer can specify the state information that should be synchronized across devices as well as the suitability of panels to different types of devices. An \emph{optimization algorithm} distributes panels to devices that maximize their match for the developer's intent.

\subsubsection{Polychrome}

Polychrome is a web application framework for creating web-based collaborative visualizations that can span multiple devices. It was developed by Badam et al.~\cite{polychrome2014}. It supports co-browsing new web applications as well as legacy websites with \emph{no migration costs}, an \emph{API to develop new web applications} that can synchronize the UI state on multiple devices to support synchronous and asynchronous collaboration and \emph{maintenance of state and input events} on a server to handle common issues with distributed applications. Polychrome provides the interaction and display space distribution mechanisms to create new collaborative web visualizations that utilize multiple devices and it provides framework modules to store the user interaction. Combined with the initial state of the website, the interaction logs are useful for synchronizing devices within the collaborative environment, consistency management and interaction replay.

\subsection{Summary}

In the sections above, we have described several different frameworks for developing cross-device applications. Some of them are only useful for specific types of applications, while others are suited for cross-device applications of all types. Despite this abundance of frameworks available, many of those frameworks provide no support for testing and debugging the applications that are developed with them. DireWolf, Polychrome and XD-MVC all include no features for testing and run in an unmodified browser. Thus, they could all benefit from a web-based tool for testing cross-device applications. Panelrama applications also run in an unmodified browser. The developers of Panelrama mention that a tool for simulating device configurations and previewing the distribution was built on request of some developers. No further details about the tool were mentioned, but the fact that developers requested such a tool shows that it is often difficult to imagine what an application would look like in different device configurations. Thus, the developers of Panelrama applications would certainly also benefit from a cross-device testing tool. Connichiwa requires the installation of a helper application, but the application itself still runs in an unmodified browser. Thus, applications developed with Connichiwa could also be tested with a web-based cross-device testing tool.

Some of the frameworks above allow the emulation of devices as well as the connection of real devices. Emulating multiple devices at a time as well as connecting real devices is a crucial feature for testing and debugging cross-device applications, but without any additional tools that help with debugging, finding and fixing bugs in the applications is still a difficult task. The framework that provides the most support for testing so far is probably XDSession, but as described above, its mechanisms are still not enough for successfully debugging cross-device applications at the source code level.
\chapter{Requirements}

In the previous chapter, we have described some tools that provide debugging mechanisms that would also be useful for cross-device testing and debugging. However, most of those mechanisms need to be extended to fulfill the needs of a cross-device testing and debugging tool. In the following sections, we will describe the key requirements for such a tool, i.e. XDTools.

\section{Emulation of Multiple Devices}

Device emulation is very popular for testing responsive web applications. The number of physical devices that a developer has access too is typically rather limited and testing on emulated devices helps cover a wider range of devices. In its simplest form, device emulation is just resizing a browser window so it looks and feels like a device with a different resolution. However, manually resizing a browser window such that it has the exact resolution of a real device is difficult. Furthermore, just changing the screen size does not realistically emulate a real device: Mobile devices typically use touch interactions, often have poor network connectivity and have access to a wide range of sensors. Those limitations have led to the emergence of more sophisticated tools for device emulation. Advanced device emulation tools such as the Device Mode in Chrome DevTools emulate different screen sizes, touch capabilities, network conditions, as well as location and acceleration sensors. They also typically provide a list containing some existing devices for the developer to choose from. In addition, the developer can create custom devices. However, all such tools have one limitation in common: They can emulate only one device per browser tab or window and even if multiple browser windows are used, those browser windows share the same local resources such as local and session storage. This limits the usefulness of such tools for cross-device application testing. Cross-device applications typically run on multiple independent devices simultaneously and thus should not share any local resources. In practice, developers employ a number of different mechanisms to prevent the sharing of local resources: First, multiple different browser profiles can be used. Second, additional browser windows can be opened in the incognito mode provided by most browsers. Lastly, multiple independent browsers can be used. However, those solutions all have some limitations: Using multiple independent browsers obviously limits the number of devices that can be emulated to a rather small number and requires the installation of all those browsers. Also, all those solutions require the developer to open multiple windows simultaneously and arrange them on the screen. This is tedious and frequent switching between browser windows is required. Furthermore, tasks like creating multiple browser profiles are time-consuming and might not be what the developer wants. Finally, not all browsers support multiple browser profiles, limiting the number of browsers on which an application can be tested. 

But not only the process of emulating devices is tedious: If the developer actually wants to use browser debugging tools, those tools have to be opened for each window separately, requiring additional screen space. The screen space of the developer's machine can also be a limiting factor in other scenarios: If the screen has a full HD resolution but the developer wants to emulate a full HD device as well as some mobile devices simultaneously, those devices cannot be ordered such that all devices are visible at the same time. The developer would need to put one window in full screen and switch browser windows when they want to access the other devices. Constantly switching between browser windows can be tedious and the consequences of interactions performed on one device cannot be observed in real-time on other devices because switching between windows requires some time. Also, emulating resolutions larger than Full HD, e.g. a 4k TV, is even more difficult.

Those limitations all contribute significantly to the difficulty of cross-device application testing. Using these limitations, we gathered a number of requirements for device emulation in XDTools:
\begin{itemize}
	\item It should be possible to \emph{emulate multiple devices} in one browser window.
	\item The emulated devices should \emph{not share any local resources}. The solution for preventing the sharing of local resources should be \emph{scalable and robust}.
	\item The screen size should not be a major limiting factor concerning the number of devices that can be emulated simultaneously. This can be put into practice by \emph{scaling devices} down without changing their resolution.
	\item The developer should be able to \emph{dynamically change the resolution} of emulated devices.
	\item The developer should have access to a list containing some \emph{existing devices}.
	\item The developer should be able to add and save \emph{custom devices}.
\end{itemize}

Scaling devices down while keeping the resolution and resizing devices are two different concepts. The difference between the two is illustrated in Figure~\ref{fig:difference_resizing_scaling}, using the W3Schools\footnote{\url{http://www.w3schools.com/}} website.

\begin{figure}[H]
  \centering
    \includegraphics[width=1.0\textwidth]{images/difference_scaling_resizing.pdf}
	\caption[Difference between resizing and scaling a device]{Difference between resizing and scaling a device}
	\label{fig:difference_resizing_scaling}
\end{figure}

\section{Easy Integration of Real and Emulated Devices}

Emulating devices is a versatile tool for testing applications on many different devices. However, it does not completely eliminate the need for testing on real devices: Device emulation is always limited to certain aspects that are being emulated, e.g. screen size, resolution, touch interactions and location. However, not every little detail of a real device can be emulated accurately. The following list provides an overview of some other limitations regarding testing on emulated devices:
\begin{itemize}
	\item Touch interactions: Even though advanced device emulation tools can also emulate touch interactions, performing a gesture with the mouse will never feel the same as the actual touch interaction. An interaction that works great with the mouse might feel awkward when performed on a real device and vice-versa. Also, multi-touch interactions such as pinching are difficult to simulate on an emulated device without real touch support.
	\item Interrupts: While using an application on a real device, the user might be interrupted by many different things, e.g. the arrival of a text message. Those interrupts cannot be realistically simulated on an emulated device.
	\item Performance: A desktop PC typically has much more computing power than a mobile device. If an application performs poorly on mobile devices, it might not even be noticed if the developer only tests on emulated devices.
	\item Display: The display quality and thus also the look of an application varies greatly depending on the device. Only emulating devices on a desktop PC cannot account for those differences in display quality. 
	\item Sensors: Modern devices have a large number of different sensors that cannot all be emulated realistically. One particular problem is the orientation of the device: A user might switch between landscape mode and portrait mode on purpose or accidentally at multiple points. Although the orientation of emulated devices can also be switched, this does not accurately simulate the behavior of a real user.
\end{itemize}
Although this list gives a good overview of the limitations of testing on emulated devices only, it is by far not complete and what happens on a real device cannot be foreseen by testing on emulated devices. Thus, testing on real devices is crucial for the successful development of an application. 

The importance of testing on real devices leads to a new requirement for XDTools: It should easily be possible to \emph{connect real devices} to XDTools. 

\section{Easy Switching of Device Configurations}

Cross-device applications are typically used by different groups of users and thus also different devices. Even the same user may sometimes use their mobile phone and laptop simultaneously and at other times only their mobile phone or tablet. Thus, the number of devices that are used simultaneously and those devices' characteristics may vary greatly. Depending on the devices connected to an application, the UI distribution and the behavior of the application might differ. Some cross-device applications are targeted to specific scenarios, e.g. a presentation room with multiple big screens that are always present, as well as some mobile devices that are only in the room when their owner is attending a presentation. In such a scenario, the developer would probably want to have access to the two static devices whenever they are testing the application and dynamically add some mobile devices. However, in other scenarios there might be no static devices at all, e.g. a public place where all visitors can connect their devices to an application related to the place. Due to the large number of different device scenarios that could be used with a cross-device application, it is a key requirement for XDTools that the developer can quickly \emph{create multiple different device configurations} and \emph{easily switch} between these device configurations. 

\section{Integration with Debugging Tools}

Many of the features integrated into the debugging tools of browsers are also immensely useful for debugging cross-device applications, and some might even be more useful than when debugging traditional web applications. However, debugging multiple devices simultaneously with those tools is difficult. We have established before that XDTools should allow emulation of multiple devices in the same browser window. While this already simplifies the debugging of multiple devices somewhat, it also introduces some new difficulties. The messages that are logged from the emulated devices are all shown in the browser debugging tools of the same window. This aggregation of logging messages is useful for seeing the messages from all devices at one glance, but identifying the device that a message came from is more difficult. The same limitation applies to JavaScript errors that are shown in the console but cannot easily be related to a device. Also, trying out things in the console by sending commands to devices becomes more difficult. Google Chrome allows the developer to switch between different frames in the console and thus address different frames with commands, but it is not always obvious which frame corresponds to which device. Also, the developer might want to try out the same thing on multiple devices and would have to switch between multiple frames to address all of them. Further limitations of the browser-included debugging tools include that navigating to the HTML of an emulated device can be rather tedious, that CSS can only be applied to one device at a time and that function breakpoints can only be added on one device at a time. Especially the last limitation can make cross-device application testing difficult because not all devices have the same responsibilities and a particular JavaScript function might not be called on all devices. Consequently, adding a breakpoint inside a function on one device might not help with debugging the function at all, if the function is not called on that device. The existence of real devices complicates things even more. By connecting the device to the PC via cable, the developer gets access to remote debugging, but debugging multiple devices at the same time gets even more complicated. The remote debugging of each device is opened in a new window, thus the developer once again has to navigate between multiple windows, something that we wanted to avoid by emulating multiple devices in one browser window. Also, getting an overview of the logging messages from all devices and sending commands to multiple devices becomes even more difficult. However, integrating existing debugging tools into XDTools is clearly desirable: Most of those tools have been around for quite some time. Thus, they are already well tested and have gone through a series of improvements. Also, almost all web developers have already used those tools for extensive testing and are already familiar with them. However, XDTools needs to extend those tools to support debugging on multiple devices simultaneously. Using all this information, we derived the following requirements for XDTools:
\begin{itemize}
	\item Logging messages from all devices should be \emph{aggregated in one place} and the device the messages originated from should be \emph{easy to identify}.
	\item JavaScript errors from all devices should be \emph{aggregated and easily identifiable}.
	\item \emph{Sending JavaScript commands to multiple devices at a time} should be possible and easy to achieve. 
	\item It should be easy to \emph{inspect the HTML of specific devices}.
	\item It should be possible to \emph{add CSS to multiple devices} at the same time (see Figure~\ref{fig:css_aggregation}).
	\item The developer should be able to \emph{add breakpoints to multiple devices} simultaneously.
	\item If technically feasible, all of the above requirements should be applied to both \emph{real and emulated devices}.
\end{itemize}

\begin{figure}[H]
  \centering
    \includegraphics[width=1.0\textwidth]{images/css_aggregation_4.pdf}
	\caption[CSS editor aggregation]{CSS editor aggregation}
	\label{fig:css_aggregation}
\end{figure}

\section{Automatic Connection Management}

In order to use a cross-device application with multiple devices simultaneously, those devices need to be paired with each other. The mechanisms for pairing devices differ between the various cross-device application frameworks: With some frameworks, all devices that open the cross-device application are paired implicitly. In other frameworks, e.g. XD-MVC, devices can be paired by copying the URL from one device to the other devices. Other frameworks have more complicated mechanisms for connecting devices. In Connichiwa, one device runs a local web server and uses Bluetooth to detect nearby devices. The device then sends the IP of the web server over Bluetooth, enabling the other devices to access the received IP in a web browser. All of those mechanisms have one thing in common: Devices are connected by opening a specific URL in the browser. However, other ways of connecting devices are also feasible: Some frameworks provide a function that can be called from one device to connect the device to another device. Also, many of the papers describing cross-device frameworks do not describe how devices are connected. Finally, cross-device applications can also be implemented independent of any framework and might use completely different mechanisms for connecting devices. Thus, it is impossible to derive all mechanisms that could possibly be used for connecting devices in a cross-device application.

However, if the developer wants to debug a cross-device application, reconnecting the devices every time a new device configuration is loaded or possibly even when devices are refreshed is tedious and time-consuming. Thus, it is desirable to have some easy way of connecting devices. From this, we derive the next requirement for XDTools: It should be possible to \emph{automatically and manually connect devices}. If possible, the provided connection mechanism should work \emph{independent of the connection mechanism} used in the application under test.

\section{Coordinated Record and Replay}

Record and replay has been used previously for recording and replaying user interactions in traditional web applications and especially AJAX web applications. The non-deterministic and asynchronous nature of web applications contributes much to the value of record and replay in web applications. When a bug is encountered in a web application, it is often difficult to determine the exact steps for reproducing the bug. Reproducing bugs becomes even more difficult in cross-device scenarios where multiple devices are involved and the interactions performed on one device have implications on other devices. Also, cross-device applications are often used by multiple users at the same time and it is difficult for one developer to simulate multiple users interacting with their devices simultaneously. Thus, we believe that record and replay can benefit cross-device application developers even more than developers of traditional web applications. However, precise mechanisms for timing the replay are needed if we want to simulate multiple users simultaneously. Also, simply replaying interactions is not enough: If something goes wrong in the application, we need some way of pausing the replay and inspecting the state of the devices. XDTools should implement a record and replay mechanism that fulfills the following requirements:
\begin{itemize}
	\item All \emph{interactions} with the device should be \emph{recorded}. 
	\item It should be possible to \emph{replay a sequence of interactions on a different device} than the device that recorded the sequence.
	\item It should be possible to \emph{pause the replay} of the interaction sequence.
	\item \emph{Accurately configuring the timing} of replays should be possible.
	\item The developer should be able to \emph{store event sequences} for later use, e.g. for testing them again in future iterations of the application under test.
\end{itemize}
\chapter{XDTools}

XDTools is an integrated set of tools that can be loaded directly into the browser. XDTools allows developers to test and debug their cross-device applications in multiple ways. The following sections will describe XDTools in detail.

\section{Overview of Features}

In Figure~\ref{fig:complete}, the complete interface of XDTools except for record and replay can be seen.

\begin{figure}[H]
  \centering
    \includegraphics[width=1.0\textwidth]{images/screenshots/complete_labeled.png}
	\caption[Screenshot: Complete Interface]{The complete interface (without record and replay)}
	\label{fig:complete}
\end{figure}

The individual features of XDTools are labeled in the screenshot:
\begin{itemize}
	\item [a)] Buttons for adding emulated devices and showing the QR code for connecting real devices.
	\item [b)] Input field for loading a URL on all devices and button to refresh all devices.
	\item [c)] Loading and saving device configurations.
	\item [d)] Area where the devices can be positioned.
	\item [e)] Proxy of a connected real device.
	\item [f)] An emulated device.
	\item [g)] The shared JavaScript console.
	\item [h)] Function debugging.
	\item [i)] The shared CSS editor.
	\item [j)] Session management.
\end{itemize}

Figure~\ref{fig:complete_remote} shows a screenshot of the remote device that is connected to XDTools in the screenshot shown above.

\begin{figure}[H]
  \centering
    \includegraphics[width=0.8\textwidth]{images/screenshots/complete_remote_2.png}
	\caption[Screenshot: Remote Device]{The connected remote device}
	\label{fig:complete_remote}
\end{figure}

XDTools also provides some options for disabling individual features, such as the shared JavaScript console. If a feature is disabled, the respective interface component is simply hidden. The options also show all stored custom devices, device configurations and event sequences and allow the developer to delete any of them. A screenshot of the options can be seen in Figure~\ref{fig:options}.

\begin{figure}[H]
  \centering
    \includegraphics[width=0.9\textwidth]{images/screenshots/options.png}
	\caption[Screenshot: Options]{Options of our main application}
	\label{fig:options}
\end{figure}

Devices can also be activated and deactivated. If a device is inactive, it is not included for the following features:
\begin{itemize}
	\item Shared JavaScript console
	\item Shared CSS editor
	\item Function debugging
\end{itemize}
All those features perform some interactions with the devices automatically, thus they should be deactivated for inactive devices. Other features like record and replay still include inactive devices because manual interaction is needed anyway before something happens with the device. Devices can be activated and deactivated by clicking on their name/ID that is displayed above the JavaScript console. When the device is active, the background of its name is in the color of the device, otherwise it is grey and the text color is in the color of the device. Figure~\ref{fig:active_inactive} illustrates the difference between active and inactive devices. The first device is inactive, the other two devices are active.

\begin{figure}[H]
  \centering
    \includegraphics[width=0.8\textwidth]{images/screenshots/active_inactive.png}
	\caption[Screenshot: Active/inactive devices]{Active and inactive devices}
	\label{fig:active_inactive}
\end{figure}

\section{Emulation of Multiple Devices}

XDTools allows the developer to emulate multiple devices simultaneously. The developer can either select the emulated devices from a list of existing devices or create a custom device. The developer can create a custom device just for one-time use, or they can save it to the list of existing devices for later use. This allows developers to quickly extend the devices they have access to with new devices. Devices can be assigned to one of four categories:
\begin{itemize}
	\item Desktop devices: For simplicity, this category includes desktop PCs as well as laptops.
	\item Tablets
	\item Mobile phones
	\item Wearables
\end{itemize}
Although some devices may not easily be classified, e.g. 2-in-1 devices that can either be used as a tablet or laptop by just plugging in a keyboard, those categories give a rough overview of the different types of devices. This categorization of devices also makes it easier for developers to emulate different types of devices without having to know any exact device names. By just adding some devices from each category, the developer can make sure that a large range of devices is covered, as devices in the same category typically have similar properties such as screen size and input modalities. Our list of existing devices includes multiple devices from each of those categories. For the desktop devices, we just provide some typical resolutions of desktop PCs and laptops. The list for tablets and mobile phones includes most of the well-known devices in that area. The list of wearables so far only includes some smart watches, as most other wearables do not have access to a modern web browser (yet). Figure~\ref{fig:adding_emulated} shows the menu for adding emulated devices.

\begin{figure}[H]
  \centering
    \includegraphics[width=0.8\textwidth]{images/screenshots/adding_device_predefined.png}
	\caption[Screenshot: Adding emulated devices]{Adding an emulated device}
	\label{fig:adding_emulated}
\end{figure}

Once the developer has created an emulated device (see Figure~\ref{fig:emulated_device}), they can move it to the desired location on the screen. Instead of ordering devices automatically, we chose to give the developer the freedom to choose how to place the emulated devices. This makes it easy to accurately simulate specific scenarios, e.g. a presentation room where two large screens are placed next to each other. To conquer the challenge of limited screen size, all emulated devices can be scaled up and down. Scaling a device does not change the resolution of the device and thus has no influence on the look of the application. This allows the developer to scale down an emulated device and have more space available for devices. If the developer has difficulties performing an interaction on a device because it is scaled down, they can just scale it up again. However, dynamically changing the resolution of an emulated device is also possible. Thus, the developer can continuously increase or decrease the resolution of the device to immediately see what the application looks with different resolutions. Finally, the developer can also change the layer of the device. Changing the layer of a device essentially moves the device to the front or back, thus devices can overlap each other and the developer can move the device that they want to use to the front.

\begin{figure}[H]
  \centering
    \includegraphics[width=0.8\textwidth]{images/screenshots/emulated_device_3.png}
	\caption[Screenshot: Emulated device]{An emulated device}
	\label{fig:emulated_device}
\end{figure}

Apart from its unique ID, each device also has a unique color. The border of the emulated device is colored with this color and the color is used in multiple other places for identifying the device. The devices also have a settings menu. In the settings menu, the developer can configure the following things:
\begin{itemize}
	\item The URL of the device.
	\item The scaling of the device.
	\item The orientation of the device.
	\item The layer of the device. 
	\item The device can be refreshed.
	\item The developer can inspect the HTML of the device.
	\item The developer can connect the device to another device by choosing it from a dropdown menu.
\end{itemize}
The device's settings are not constantly used by the developer and showing them at all times would occupy valuable screen space. Thus, the setting menu can be extended and collapsed by clicking a button. An example of a settings menu can be seen in Figure~\ref{fig:settings_menu}. 

\begin{figure}[H]
  \centering
    \includegraphics[width=0.8\textwidth]{images/screenshots/settings_menu_2.png}
	\caption[Screenshot: Settings menu emulated device]{Settings menu of an emulated device}
	\label{fig:settings_menu}
\end{figure}

Finally, XDTools also includes a mechanism that prevents the sharing of local resources between emulated devices.
 
\section{Easy Integration of Real and Emulated Devices}

QR codes have become increasingly popular over the last few years and almost all devices nowadays are equipped with at least one camera. Thus, XDTools includes a QR code that can be displayed and scanned by developers to connect a real device to XDTools. As a fallback mechanisms, devices can also be connected by typing a URL in the browser of the device. This makes it easy and efficient to connect a large number of real devices to XDTools. When the URL is opened, the device loads the application under test. The real devices only show the application under test and no interface elements. The developer can use the application on the real device, but everything relating to testing and debugging is coordinated through the developer's machine. Each connected real device is represented by a proxy within XDTools. The proxy of the real device also contains a settings menu similar to the one of an emulated device. However, the settings menu is missing a few things compared to emulated devices:
\begin{itemize}
	\item Scaling of the device: There is no need to scale real devices, as no \lstinline|iframe| is shown in the main application.
	\item Switching orientation: The orientation of a real device can be switched on the real device itself by simply rotating the real device.
	\item Inspecting the HTML: The developer's machine has no access to the HTML of the real device, thus it cannot be inspected.
\end{itemize}
Figure~\ref{fig:settings_menu_remote} shows the settings menu of a remote device. The settings menu only contains interface elements for setting the URL of the device, refreshing the application under test, changing the layer of the device and connecting the device to other devices.

\begin{figure}[H]
  \centering
    \includegraphics[width=0.5\textwidth]{images/screenshots/remote_device.png}
	\caption[Screenshot: Settings menu remote device]{Settings menu of a remote device}
	\label{fig:settings_menu_remote}
\end{figure}

The proxy of the real device can be moved around just like the emulated devices and its settings menu can also be collapsed and extended. Furthermore, each real device also has a unique ID and color for easy identification.

The developer can use emulated devices alongside real devices. Thus, XDTools allows the developer to test their application on only emulated devices, only real devices, or both at the same time. This flexibility makes it easy to test a large number of different scenarios.

\section{Easy Switching of Device Configurations}

XDTools allows the developer to create a device configuration, save it for later, and then re-use it. A device configuration consists of the following information:
\begin{itemize}
	\item The number of emulated devices.
	\item The types of the devices.
	\item The position and scaling of the devices.
\end{itemize}
The developer can easily re-use device configurations without much effort and switch between different device configurations efficiently. Thus, testing a cross-device application in many different device scenarios can be done without much effort. A user can just load one device configuration, try out the application and switch to the next device configuration if everything works as expected. The developer can also create a device configuration where only the static devices are saved in the configuration, e.g. in the example mentioned above, the developer could create a device scenario where only the big screens are already configured and then dynamically add some mobile devices to create a more realistic scenario.

For saving device configurations, the developer can type the name of a device configuration into an input field and then either click a save button to save the current device configuration under that name or click a load button to load a device configuration with that name (if it exists). Autocomplete is used for providing suggestions for existing device configurations that could be loaded (see Figure~\ref{fig:session_management}). 

\begin{figure}[H]
  \centering
    \includegraphics[width=0.6\textwidth]{images/screenshots/session_management_2.png}
	\caption[Screenshot: Saving/Loading device configurations]{Saving/Loading Device Configurations}
	\label{fig:session_management}
\end{figure}

\section{Integration with Debugging Tools}

For tighter integration with the browser's debugging tools, we included some popular tools that are already well-known from the debugging tools.

\subsection{Shared JavaScript Console}

XDTools includes a JavaScript console that is shared between all emulated and real devices. Each emulated and real device forwards all logging messages to the console. The console then aggregates all console outputs. Each message is displayed in the color of the device that it was sent from. This color-coding makes it trivial to identify the device a message originated from. Apart from forwarding logging messages, we also forward JavaScript errors. Whenever a JavaScript error occurs, the JavaScript error message is sent to the console, together with the stack trace if available. When a JavaScript error with stack trace is received by the console, the stack trace is split into multiple lines where each line represents one entry of the stack trace. By default, the developer only sees the error message itself without the stack trace. By clicking on a down-arrow button next to the error message, the developer can extend and collapse the stack trace. Figure~\ref{fig:stack_trace} shows what such a stack trace looks like in our JavaScript console.

\begin{figure}[H]
  \centering
    \includegraphics[width=1.0\textwidth]{images/screenshots/stack_trace.png}
	\caption[Screenshot: Stack trace]{Stack trace}
	\label{fig:stack_trace}
\end{figure}

For easier identification of the type of a message, all messages are assigned to one of four categories:
\begin{itemize}
	\item Warnings: This includes all warning logging messages.
	\item Errors: This includes all error logging messages and JavaScript errors.
	\item Info: This includes all info logging messages.
	\item Log: This includes all other types of messages.
\end{itemize}
Depending on the type of message, a different symbol is used in front of the message when it is displayed in the shared JavaScript console. This makes it easy to see if a message is an error or just a simple logging message.

The large number of messages that are displayed in the console due to the aggregation of many devices makes it a key requirement that there are some ways of filtering messages. The messages in the console can either be filtered by the type of message, e.g. show only all error messages, or by text. If the console messages are filtered by text, all messages that do not contain the text to filter by anywhere are hidden. If the filter is removed, the hidden messages are shown again. If a device is deactivated, it does not forward any messages or errors anymore and all existing messages are filtered out until the device is activated again. 

Our console also allows sending commands to multiple devices. If the developer wants to send a command to the devices, they can type the command into an input field below the console and press the enter key. The developer can either send the command to all devices or they can deactivate some devices and only send the message to a subset of devices. Being able to send a command only to a subset of devices is important in a cross-device setting because some commands might only make sense on some devices and executing them on all devices could lead to potential errors or a decrease in performance. The return values of commands are also displayed in the console, again color-coded to match the device they came from. Thus, otherwise complicated tasks like checking the value of a global variable become trivial. Our custom console works exactly the same for emulated as well as real devices. This aggregation of console outputs from both emulated and real devices at least partially eliminates the need for remote debugging.

The shared JavaScript console also has a history function similar to the console in Chrome DevTools. By using the arrow keys, the developer can navigate through the history and call previous commands again.

The screenshot in Figure~\ref{fig:js_console} shows the shared JavaScript console including a few examples of how to use it. First, the developer tries to call a function on all devices but makes a typo. Because the function with the typo in the name does not exist, an error message is displayed by all devices. The developer then notices the typo, types the correct function and sees the return values of the function (the coordinates of a cinema). Finally, the developer wants to see which roles have been assigned to the devices. They type the name of the variable that contains the roles and the value of this variable on all devices is shown in the console.

\begin{figure}[H]
  \centering
    \includegraphics[width=1.0\textwidth]{images/screenshots/js_console_2.png}
	\caption[Screenshot: JavaScript console]{Shared JavaScript console}
	\label{fig:js_console}
\end{figure}

\subsection{Function Debugging and Inspection}

Function debugging allows developers to debug a function on all emulated devices or a subset of emulated devices without having to add breakpoints to each device individually.

In an area at the bottom of the interface of XDTools, the developer has access to an input field where they can type the name of the function they want to debug.  Below the input field, the list of functions that are currently debugged is shown. Whenever one of the debugged functions is called on any device, the debugger pauses at the beginning of the function and the developer can perform their debugging actions. The device on which the function was called is highlighted until the function call returns. This makes it easy to identify the device that is currently being debugged.

Figure~\ref{fig:function_debugging_complete} shows a screenshot while a function is being debugged. In the screenshot, the highlighted device, the Chrome DevTools with the debugged function and the list of debugged functions can be seen.

\begin{figure}[H]
  \centering
    \includegraphics[width=1.0\textwidth]{images/screenshots/function_debugging_complete.png}
	\caption[Screenshot: Function debugging]{The complete interface while debugging a function}
	\label{fig:function_debugging_complete}
\end{figure}

If the developer is done debugging a function, they can remove it from the list of debugged functions.

The developer can also just inspect the function without it being called by clicking on a button next to the function name in the list of debugged functions. If this button is clicked, an emulated device is picked at random and the source code of the function is opened on this device. 

Unfortunately, function debugging only works on emulated devices. Chrome's DevTools do not have access to the JavaScript code of a connected real device, thus debugging the function using the DevTools is impossible.

\subsection{HTML Inspection}

XDTools allows the developer to directly jump into the HTML of an emulated device. Inside the settings menu of each emulated device, a button can be clicked to inspect the HTML of the device. Clicking this button opens the \lstinline|body| element of the device in the Chrome DevTools. 

HTML inspection is only available on emulated devices for the same reasons that we already mentioned before.

\subsection{Shared CSS Editor}

XDTools also includes a custom CSS editor. The CSS editor is designed to feel similar to the CSS editors typically provided by browsers. Thus, the developer can specify a selector and then add some rules that are applied to HTML elements that match the selector. The CSS rules are applied to all active devices, including both real and emulated devices. All CSS rules can be deactivated and activated again, edited, or removed completely. This allows the developer to quickly change the CSS of multiple devices and immediately see the result on all those devices. This is considerably less effort than adding rules to all devices individually or editing the CSS file, saving it and reloading all devices multiple times.

The CSS editor also has some autocomplete functionality: If the developer starts typing the name of a property, the property is automatically completed using the first CSS property name that matches the text typed by the developer (see Figure~\ref{fig:css_autocomplete}).

\begin{figure}[H]
  \centering
    \includegraphics[width=1.0\textwidth]{images/screenshots/css_autocomplete.png}
	\caption[Screenshot: CSS editor autocomplete]{Autocomplete in the CSS editor}
	\label{fig:css_autocomplete}
\end{figure}

Figure~\ref{fig:css_editor} shows a screenshot of the CSS editor in action. In this screenshot, some CSS rules are added to the \lstinline|body| element of all devices.

\begin{figure}[H]
  \centering
    \includegraphics[width=1.0\textwidth]{images/screenshots/css_editor.png}
	\caption[Screenshot: CSS editor]{Shared CSS Editor}
	\label{fig:css_editor}
\end{figure}

In Figure~\ref{fig:css_applied}, the effects of the CSS shown in the screenshot above are shown. 

\begin{figure}[H]
  \centering
    \includegraphics[width=0.5\textwidth]{images/screenshots/emulated_device_4.png}
	\caption[Screenshot: CSS effects]{CSS applied to emulated device}
	\label{fig:css_applied}
\end{figure}

\section{Automatic Connection Management}

The devices in XDTools can either be connected automatically or manually. Each set of connected devices represents a session. For each session, a checkbox allows the developer to toggle on or off auto-connect. When the first device is created or connected, auto-connect is turned on by default. If auto-connect is on, all newly created and connected devices will automatically be connected to that session. Thus, if the developer simply adds or connects a number of devices, they will automatically all be connected. If the developer wants to have multiple sessions, they can turn off auto-connect and then add more devices. Each device also has a drop-down menu in its settings menu for manually connecting to other devices. 

All current sessions are displayed at the bottom of the page of our application. Each session displays all connected devices (see Figure~\ref{fig:sessions}). The developer can also refresh all devices in a session and reset a session. Resetting a session assigns new IDs to all devices and thus erases the data of the devices. 

\begin{figure}[H]
  \centering
    \includegraphics[width=0.6\textwidth]{images/screenshots/session_management.png}
	\caption[Screenshot: Session]{A device session}
	\label{fig:sessions}
\end{figure}

So far, automatic connection is only possible for applications that connect via URL. However, the URL required for connecting depends on the application. XDTools provides a custom connection function that can be adjusted by the developer such that it returns the appropriate URL.

\section{Coordinated Record and Replay}

XDTools allows the developer to record a sequence of interactions and replay them. Recording can be started on one device at any time. Once recording has started, the developer can perform the desired interactions on the device. After finishing recording, the interactions can be replayed on the same device, moved to other devices, or saved for later.  Furthermore, event sequences can be cut into multiple parts. The timing of the replays can be configured arbitrarily by dragging and dropping event sequences. This allows the developer to configure replays in many different ways: They can be executed in parallel, one after another, or anything in-between. The replays can then be started on all devices simultaneously r on one device at a time. This makes it easy to simulate multiple users and devices in a cross-device environment. Since the events contained in an event sequence are performed by a real user, i.e. the developer, the timing of the individual events in an event sequence is very realistic.

After a device finishes recording, the event sequence is visualized (Figure~\ref{fig:record_replay}).

\begin{figure}[H]
  \centering
    \includegraphics[width=1.0\textwidth]{images/screenshots/record_replay.png}
	\caption[Screenshot: Record and replay]{Record and Replay}
	\label{fig:record_replay}
\end{figure}

Furthermore, the developer can add breakpoints to the timing of the replay. Next to the event visualizations, a timeline is shown as well as a narrow empty column where the developer can click to add a breakpoint at a certain point in time. As soon as all events that occur before this point of time have been replayed, the replaying pauses and the developer can inspect the state of the devices. If a breakpoint is reached, the breakpoint that was reached is highlighted. After the developer chooses to continue, replaying is continued until the next breakpoint is reached. After the last breakpoint has been reached, the event replaying continues until all events have been replayed. The developer can also add breaks of one second between events to delay all following events. This allows the developer to spend some more time looking at the application before the replaying continues without having to set breakpoints. 
\section{Architecture}

\section{Choice of Technologies}

\section{General Features}

\section{Device Emulation}

\section{Connecting Real Devices}

\section{Shared JavaScript Console}

\section{Function Debugging}

\section{Shared CSS Editor}

\section{Record/Replay}
\section{XDCinema}

\section{XDYouTube}

\section{Insights}

\section{Setup}

The study was carried out in a room of the GlobIS group. The participants were sitting in front of a 27-inch screen with a 2560x1280 and had access to an English (US) keyboard and a mouse. They also had access to two real devices: A Nexus 7 and an HTC M9. Furthermore, they were given a tutorial sheet about Javascript that contained some functions that are useful for DOM modification as well as arrays. During the study, the instructor was sitting next to the participants and was available to answer any questions that occurred during the study. 

\subsection{Participants}

We recruited 12 participants that all were university members in the department of computer science at ETH Zurich. Most participants were either PhD or Master students, but there were also some Bachelor students. It was required that all participants have at least basic knowledge about front-end web technologies (i.e. HTML, CSS, Javascript). The age of the participants ranged from 23 to 33 and the median age was 26. 

\subsubsection{Previous Experience}
We asked all participants about their previous experience with web application development and Javascript in particular, as well as about their previous experience with responsive web applications and cross-device web applications. Furthermore, we asked them about whether they have used Chrome DevTools before and how they used them. The participants rated their skills in web application development and Javascript on a 5-point Likert scale (see Figure ~\ref{fig:xp}) from basic to proficient and also gave the numbers of years in experience (see Figure ~\ref{fig:years_of_xp}) that they had in web application development and Javascript.

\begin{figure}[H]
  \centering
    \includegraphics[width=0.8\textwidth]{images/charts/xp.pdf}
	\caption{Previous experience}
	\label{fig:xp}
\end{figure}

\begin{figure}[H]
  \centering
    \includegraphics[width=0.8\textwidth]{images/charts/years_of_xp.pdf}
	\caption{Years of experience}
	\label{fig:years_of_xp}
\end{figure}

Eight out of the twelve participants stated that they already had some experience with developing responsive web applications. All of them either used browser tools for emulating devices to test their applications or real devices. Some also used both.

Seven participants already had some experiences with cross-device application development. Again, most of them used browser tools for emulating devices or real devices for testing their applications. Four of them either used multiple browsers, multiple browser profiles or incognito modes to simulate multiple devices on one device. Thus, about half of the participants that already had some cross-device experience constantly used multiple devices to test their applications.

Most of the participants already had experience with Chrome DevTools, only three participants indicated that they had never used them before. We asked participants how often they used certain features of Chrome DevTools, in particular Device Mode, HTML and CSS inspection, Javascript debugging and the console (see Figure ~\ref{fig:devtools_xp}). Device Mode was rarely used by the participants, which is no surprise, given that it is a rather new feature. All participants stated that they often use HTML and CSS inspection, thus this seems to be the most popular feature. The console was also used rather often. Surprisingly, Javascript debugging was not that popular, less than half of the participants stated that they often use it.

\begin{figure}[H]
  \centering
    \includegraphics[width=0.8\textwidth]{images/charts/devtools_xp.pdf}
	\caption{Previous experience with Chrome DevTools}
	\label{fig:devtools_xp}
\end{figure}

\subsection{Tasks}

We used two different applications for the tasks. For each application, there were two tasks; one was about finding and fixing a bug in the code, and the other one was about implementing a new feature. The maximum time for the tasks where the participants fixed a bug was 15 minutes and the maximum time for implementing a feature was 30 minutes. After this time, we aborted the task unless it was clear that the participants would finish within the next 2 to 3 minutes. Each participant had to complete all four of the tasks; the tasks of one application with our tools, the tasks of the other application without them. The order of the applications as well as whether the first two tasks were with or without the tools was random. One of the applications was a cross-device YouTube application, called XDYouTube. The other application was a cross-device cinema application, called XDCinema.

\subsubsection{XDYouTube}
XDYouTube allows users to use their personal devices to search for videos and add them to a queue. The videos from the queue were then played one after the other on the largest of the devices. Users could also see the title and description of the currently playing video as well as the videos that are still in the queue by switching their device into landscape mode.

The first task with XDYouTube was to fix a bug concerning the video queue. As soon as one video finishes playing, the next video was dequeued from the queue and started playing. However, when no video was in the queue, a Javascript error occurred and caused the next video that was added to the queue not to play. The users were given a description of the task and then had to reproduce and fix it. 

For the second task, we asked participants to implement a remote control that could play and pause the current video. The participants had to implement two functions: One was called when the remote control button was clicked (the button as well as the event handler were already implemented), the other was called when a shared variable that states whether the video is paused or playing is changed. Thus, the participants had to change the shared variable whenever the button was clicked and to react accordingly on all devices if the shared variable changes, i.e. they had to pause or play the video on the device that plays the video and they had to change the text of the remote control button on all other devices. Furthermore, they had to change the CSS of the button such that it looked similar to a picture of a button that was given to them.

\subsubsection{XDCinema}

XDCinema allows users to search for a city and date on one device. The device then shows a list of movies that play in this city on that date as well as the cinemas where the movie is played and the time that the movie starts. If the user clicks on a cinema, a summary of the movie as well as other information about the movie is shown on another device. If the user clicks on a cinema, the location of the cinema is shown on another device.

The first task was to fix a bug where the location of most cinemas was displayed wrongly, even though the information in the database is correct. The bug was that in one function, "j" was used instead of "i", which caused a wrong location to be returned.

In the second task, the participants first had to complete the implementation of a function that shows the prices of each cinema where the movie plays below the description of the cinema. A skeleton for this function was already given where a loop over all cinemas that show the movie was already implemented, the participants only had to fill in the body of the loop. The second part was about first highlighting the correct price when the user clicks on a cinema in the search view and improving the CSS for highlighting. 

\subsection{Evaluation Methods}

\subsubsection{Questionnaires}
At the beginning of the study, each participant had to fill out a questionnaire about their background information. After every task, the participant had to fill out another questionnaire with the following questions:
\begin{itemize}
	\item It was easy to complete the task with the tools I had access to.
	\item I felt efficient completing the task with the tools I had access to.
	\item It was challenging to complete the task with the tools I had access to.
	\item The tools I had access to were well suited for completing the task.
\end{itemize}
The questions could be answered on a 5-level Likert scale from "Strongly Disagree" to "Strongly Agree". In the tasks where the participants had access to our tools, we also asked them to rate the usefulness of the individual features of the tool on a 5-level Likert scale.

After completing all tasks, the participants had to fill out a final questionnaire where they could answer some questions that compare our tools to the usual Chrome browser tools. They had to answer the following questions:
\begin{itemize}
	\item Did you find it easier to debug with or without the tool?
	\item Did you feel more efficient debugging with or without the tool?
	\item Did you prefer debugging with or without the tool?
	\item Did you find it easier to implement a feature with or without the tool?
	\item Did you feel more efficient implementing a feature with or without the tool?
	\item Did you prefer implementing a feature with or without the tool?
\end{itemize}

In addition, they also answered some general questions about our tool?
\begin{itemize}
	\item It was easy to learn how to use the tool.
	\item I felt confident using the tool.
	\item The tool was unnecessarily complex.
	\item The tool would be useful for debugging cross-device applications.
	\item The tool would be useful for implementing cross-device applications.
	\item I would use the tool for debugging cross-device applications.
	\item I would use the tool for implementing cross-device applications.
\end{itemize}
Those questions could again be answered on a 5-level Likert scale from "strongly disagree" to "strongly agree".

Finally, the participants could state which features of the tool they would use for debugging and implementing cross-device applications and they could also write some comments about the tool if they wanted to.

\subsubsection{Video Recording}
In addition to letting participants fill out questionnaires, we also used a video camera to record the participants while completing the tasks. This was mainly done to make sure that no important information was lost and so some strategies for solving tasks could be extracted from the videos. 

\subsubsection{Personal Feedback}
At the end of the study, participants were encouraged to share any comments that they still wanted to mention and to give their opinion about the tool. Any comments that the participants had given during the study were also noted.

\subsubsection{Time Measuring}
For each participant, the time required for completing each task was measured. This was mainly done to detect any major discrepancies between completion times with and without the tool. However, exact times are not considered relevant for evaluation because they highly depend on the participant and on the hints given by the instructor during the study.

\section{Results}

In the following sections, we will present the results from the individual tasks as well as the more general results. For each task, we will compare how people answered the questions in the per-task questionnaires with and without our tools. 

\subsection{XDCinema: Fixing a Bug}

The results for the task where the participants had to fix a bug in XDCinema can be seen in Figure ~\ref{fig:xdc_bug_comparison}. The figure shows the median values for the questions asked after the task with and without our tools. For the question that asks about how challenging it was to complete the task with the tools the participant had access to, a lower value is better; for all other questions, a higher value is considered better. The median values for the suitability of the tools has a rather big difference, while the differences are smaller for the other tasks. Concerning easiness and efficiency, only a very small difference in favor of our tools can be seen. One reason for the bigger difference in the suitability could be that unlike easiness and efficiency, suitability does not really depend on the task itself. In other words, if a task is difficult, the easiness will be rated lower regardless of whether our tools are used or not. On the other hand, the suitability of the tools for the task does not change with the difficulty of the task.

\begin{figure}[H]
  \centering
    \includegraphics[width=0.8\textwidth]{images/charts/xdc_bug_comparison.pdf}
	\caption{XDCinema bug task - Comparison}
	\label{fig:xdc_bug_comparison}
\end{figure}

Figure ~\ref{fig:xdc_bug_features_used} shows how many participants used the individual features of our tools and how useful they found them. Obviously, the figure only includes the participants that had access to our tools. None of the participants used real devices and all of them used device emulation instead. However, two participants had a neutral opinion about device emulation. This could be due to the fact that the bug they had to fix is actually rather trivial and could maybe be solved faster by just looking at the code. The connection features and function debugging were used by all except one participant and were very appreciated by the participants. The shared JavaScript console was rather unpopular for this task, probably also due to the simplicity of the task and due to the fact that the bug produced no JavaScript errors that would be displayed in the console.

\begin{figure}[H]
  \centering
    \includegraphics[width=0.8\textwidth]{images/charts/xdc_bug_features_used.pdf}
	\caption{XDCinema bug task - Features used}
	\label{fig:xdc_bug_features_used}
\end{figure}

\subsection{XDCinema: Implementing a Feature}

In Figure ~\ref{fig:xdc_impl_comparison}, the results for the implementation task in XDCinema can be seen. Again, the figure shows the median values. In this task, the difference in suitability is less pronounced than in other tasks, whereas the difference in rather large. In general, the task was considered as rather easy independent of the tools the participant had access to, although participants that had access to the tools perceived it even as a bit easier. In conclusion, it seems that the task is easy anyways, but our tools makes participants feel much more efficient when completing the task. Surprisingly, if we compare the average or median completion times for this task, the participants that had access to our tools were considerably slower. In fact, this is the only task where the difference in completion times with and without our tools is noticeable; the completion times for all other tasks are almost equivalent. However, those two facts do not necessarily contradict each other, as there were participants with very different experience levels and as it is random which participants have access to our tools and which not and the number of participants is rather low, it is possible that almost all participants with low experience fall into the same category. In general, the completion times should not be considered as especially relevant, after all the instructor also gave some hints during the study and this can also distort completion times significantly. However, it would be interesting to see how completion times differ if there is a larger number of participants. 

\begin{figure}[H]
  \centering
    \includegraphics[width=0.8\textwidth]{images/charts/xdc_impl_comparison.pdf}
	\caption{XDCinema implementation task - Comparison}
	\label{fig:xdc_impl_comparison}
\end{figure}

Figure ~\ref{fig:xdc_impl_features_used} again shows the use and the ratings of the individual features. Again, no participant used the real devices. All participants used device emulation and the connection features and in contrast to the bug task in XDCinema, device emulation was rated as useful by all participants. Function debugging was used a bit less than in the bug task, but the shared JavaScript console was used much more. It makes sense that function debugging is used more when fixing a bug; if one implements a feature and it works immediately when testing it, there is no need to debug a function, but if one has to fix a bug, there obviously must be a bug in a function and thus it makes much more sense to debug functions. The shared JavaScript console was rarely used to send commands and most participants did not use logging for solving the task, but many participants had some syntax errors when first testing the feature and noticed the error messages in the console. This also explains why the console was used more in the implementation task than in the bug task: Generally, console outputs are very useful for debugging, but in this specific bug, there were no errors in the console in contrast to the implementation task, where syntax errors were shown in the console. Finally, the CSS editor was also used by some participants in this task. However, some completed the CSS part of the task using only the CSS file. This may be because they did not think of the CSS editor at this specific moment, or because they know CSS so well that they can just write everything down immediately. 

\begin{figure}[H]
  \centering
    \includegraphics[width=0.8\textwidth]{images/charts/xdc_impl_features_used.pdf}
	\caption{XDCinema implementation task - Features used}
	\label{fig:xdc_impl_features_used}
\end{figure}

\subsection{XDYouTube: Fixing a Bug}

Figure ~\ref{fig:xdyt_bug_comparison} shows the results for the bug task in XDYouTube. The difference in suitability between our tools and the usual browser tools was most significant in this task with a difference of 2. The difference in efficiency is also rather large. Surprisingly, there is no difference in easiness and the task was also rated as almost equally challenging with and without our tools despite the large differences in the other questions. It seems that for this task, our tools did not make the task any easier to solve, but the participants felt more efficient when completing it. During the study, we noticed that most participants had problems reproducing the bug and almost all participants required some hints and finished the task more or less around the time limit. This could explain why the task was perceived as equally difficult with and without our tools: The participants did not really find the bug without help anyway, independent of whether they had access to our tools, so it makes sense that they would consider the task as difficult in general. 

\begin{figure}[H]
  \centering
    \includegraphics[width=0.8\textwidth]{images/charts/xdyt_bug_comparison.pdf}
	\caption{XDYouTube bug task - Comparison}
	\label{fig:xdyt_bug_comparison}
\end{figure}

In Figure ~\ref{fig:xdyt_bug_features_used}; the use and ratings of the individual features can be seen. All of the participants used device emulation and connection features and rated them well. Function debugging was also used by almost all participants, probably because it was difficult to reproduce the bug and the participants wanted to see what was going on in the functions. About half the participants used the shared JavaScript console, mainly to see the error produced in the function that caused the bug. One participant connected the Nexus 7 to our tools and liked the feature, but no statement about the general usefulness of the feature can be made from just one participant. 

\begin{figure}[H]
  \centering
    \includegraphics[width=0.8\textwidth]{images/charts/xdyt_bug_features_used.pdf}
	\caption{XDYouTube bug task - Features used}
	\label{fig:xdyt_bug_features_used}
\end{figure}

\subsection{XDYouTube: Implementing a Feature}

Figure ~\ref{fig:xdyt_impl_comparison} shows the results for the implementation task in XDYouTube. In this task, all questions were clearly rated in favor of our tools. While the participant answered the questions in a rather neutral way when they did not have access to our tools, they clearly stated that the task was easy to complete and felt efficient to complete with our tools. This task differs from the others a bit, because all other tasks had questions where the difference in median was 0.5 or 0, whereas the difference is at least 1 in this task for every question.

\begin{figure}[H]
  \centering
    \includegraphics[width=0.8\textwidth]{images/charts/xdyt_impl_comparison.pdf}
	\caption{XDYouTube implementing task - Comparison}
	\label{fig:xdyt_impl_comparison}
\end{figure}

In Figure ~\ref{fig:xdyt_impl_features_used}; the use and ratings of the individual features can be seen. Once again, device emulation and the connection features were used by every participant. The shared JavaScript console and CSS editor were about equally popular and rated as very useful except for one participant that had a neutral opinion on the console. Function debugging was rarely used for this task. This is rather surprising because many participants had a bug where they had switched playing and pausing the video at the beginning and this could probably have been solved easily by debugging the function. However, most participants just got stuck at the bug and required some hints to fix it instead of debugging their functions. Those participants were in the same group as the one that required significantly longer for solving the implementation task in XDCinema which could again indicate that the participants in this group had a bit lower experience in web application development. When comparing the web application development experience stated in the general questionnaire by the two groups, the other group has a median experience of 4.5 while this one has a median experience of 3.5, so there indeed seems to be some difference in experience levels.

\begin{figure}[H]
  \centering
    \includegraphics[width=0.8\textwidth]{images/charts/xdyt_impl_features_used.pdf}
	\caption{XDYouTube implementing task - Features used}
	\label{fig:xdyt_impl_features_used}
\end{figure}

\subsection{General}

In general, our tools were rated as very useful for all tasks with a median value of 5 for every task. In contrast, the usefulness of the usual browser tools was rated with median values between 3 and 4, thus we consider our tools as useful for implementing and debugging cross-device applications. The efficiency was also rated as significantly better for all tasks except the bug task in XDCinema, where the difference was only 0.5. This indicates that our tools make the process of testing cross-device applications more efficient. The difference in easiness seems to depend partially on the type of task; in the tasks where the participants had to implement a feature, completing the task with out tools was clearly considered easier, whereas the difference was smaller for the tasks where the participants had to fix a bug. Surprisingly, the results for the question about how challenging it was to complete the task do not necessarily relate to the results of the question about easiness. For the bug task in XDCinema and the implementation task in XDYouTube, our tools seem to make the task less challenging, but for the other two tasks, the differences are very minor. In the XDYouTube bug task, completing the task with our tools is even rated as slightly more challenging.

\section{Discussion}

Figure ~\ref{fig:implementing_easier} shows that about three quarters of all participants considered implementing a feature easier with our tools. Only one participant found it easier to implement a feature without our tools. However, in principle, this option is redundant as the participants still have access to the browser tools even when they have access to our tools. Thus the relevant conclusion is that about one quarter of the participants did not see any gain in easiness from using our tools. The same applies to the question about whether implementing a feature feels more efficient with our tools (see Figure ~\ref{fig:implementing_efficient}); this figure shows exactly the same results as the figure about easiness. However, all except one participant preferred implementing a feature with our tools (see Figure ~\ref{fig:prefer_implementing}). Thus, it seems that participants like to have access to our tools even if they cannot directly relate them to a decrease in difficulty or to an increase in efficiency.

\begin{figure}[H]
  \centering
    \includegraphics[width=0.6\textwidth]{images/charts/implementing_easier.pdf}
	\caption{Easiness of implementing a feature}
	\label{fig:implementing_easier}
\end{figure}

\begin{figure}[H]
  \centering
    \includegraphics[width=0.6\textwidth]{images/charts/implementing_efficient.pdf}
	\caption{Efficiency of implementing a feature}
	\label{fig:implementing_efficient}
\end{figure}

\begin{figure}[H]
  \centering
    \includegraphics[width=0.6\textwidth]{images/charts/prefer_implementing.pdf}
	\caption{Preference for implementing a feature}
	\label{fig:prefer_implementing}
\end{figure}

Figure ~\ref{fig:debugging_easier} shows that most participants found it easier to debug a cross-device application with our tools. The results get even more clear if we look at Figure ~\ref{fig:debugging_efficient} and Figure ~\ref{prefer_debugging}. Those two figures show that all participants felt more efficient when debugging with our tools and also preferred debugging with our tools.

\begin{figure}[H]
  \centering
    \includegraphics[width=0.6\textwidth]{images/charts/debugging_easier.pdf}
	\caption{Easiness of debugging}
	\label{fig:debugging_easier}
\end{figure}

\begin{figure}[H]
  \centering
    \includegraphics[width=0.6\textwidth]{images/charts/debugging_efficient.pdf}
	\caption{Efficiency of debugging}
	\label{fig:debugging_efficient}
\end{figure}

\begin{figure}[H]
  \centering
    \includegraphics[width=0.6\textwidth]{images/charts/prefer_debugging.pdf}
	\caption{Preference for debugging}
	\label{fig:prefer_debugging}
\end{figure}


We also asked the participants if they would use our tools for debugging and implementing cross-device applications and if they think that our tools would be useful. The results can be seen in figure ~\ref{fig:usefulness_tool}. Almost all participants think that our tool would be very useful for implementing as well as debugging cross-device applications and the remaining few also think that they would be useful. All participants would use our tools for implementing cross-device applications and all except one participant would use them for debugging cross-device applications.

\begin{figure}[H]
  \centering
    \includegraphics[width=0.8\textwidth]{images/charts/usefulness_tool.pdf}
	\caption{Usefulness of our tools}
	\label{fig:usefulness_tool}
\end{figure}

Finally, you can see the results for the general questions about our tools in Figure ~\ref{fig:tool_general}. The figure shows that our tools were perceived as easy to learn despite the many different features and the participants also felt rather confident using our tools. Our tools are not considered as unnecessarily complex at all. This indicates that the interface of our tools is generally well-structured and it would be easy for developers to get used to our tools.

\begin{figure}[H]
  \centering
    \includegraphics[width=0.8\textwidth]{images/charts/tool_general.pdf}
	\caption{General evaluation of our tools}
	\label{fig:tool_general}
\end{figure}

Figure ~\ref{fig:would_use_features} shows which features the participants would use for implementing and debugging cross-device applications. In general, people would rather use emulated devices rather than connecting real devices for both implementing and debugging. This is understandable as most things can be done just as well with emulated devices as with real devices. Some participants mentioned that they would not really use real devices during implementing, but that they would test their application on real devices after finishing implementing to make sure the application works fine on them. The connection features are almost unavoidable to use, thus it is surprising that some participants state that they would not use them. However, for some parts of debugging and implementing, one device might be sufficient for testing and no connection features would be required. One participant mentioned that the connection features seem very natural and that there is no point in asking about their usefulness because it is obvious that they are useful. This indicates that the feature was indeed greatly appreciated by some participants. The shared JavaScript console is equally popular for debugging and implementing and would be used by almost all participants. Function debugging is more popular for debugging than for implementing. This corresponds to the actual results of the study and has been elaborated before. Apart from connecting real devices, the shared CSS editor is the least popular. This may be due to the fact that browsers already have quite mature CSS editors and it might be possible to test CSS on one device at a time in many cases. Also, the CSS parts of our tasks were rather simply, thus the real value of such a feature might not be obvious to the participants. While things like changing the background color of a button can easily be done on just one device, more complex CSS problems like positioning elements require more effort and look much different on different devices.

\begin{figure}[H]
  \centering
    \includegraphics[width=0.8\textwidth]{images/charts/would_use_features.pdf}
	\caption{Features that the participants would use}
	\label{fig:would_use_features}
\end{figure}

During the study, some participants mentioned that they found device emulation very useful because they can have all devices in one place instead of having to manage different browser windows and profiles. Reloading all devices at once was also considered very useful. With multiple browser windows, each window has to be reloaded individually which makes this a much more tedious task. It was also appreciated that the devices are still connected after reloading, though this is also the case with multiple browser windows. The participants really liked function debugging, the only critique was that it does not show on which device a function is called which was fixed after the study. In general, the output aggregation of the shared JavaScript console was considered more useful than sending commands. One participant even mentioned that he would not use it to send commands, but the aggregation is very useful. 

Record/replay was disabled for the user study, but one participant that had attended a presentation about our tools before, mentioned that they would find it immensely useful. They consider it as a powerful feature that could be very helpful for replicating bugs in an application and for regression testing. Often, it is not clear how to reach a bug and being able to record the set of interactions that lead to the bug and then replay them can simplify this process.
\section{Future Work}

Extended record and replay: Record things like dates, server responses, random numbers, ...

Extended device emulation: Network, location, touch, ...

Tighter integration with debugging tools or browser

Long-term study: Let participants use and evaluate tools over longer period of time

\appendix

\chapter{Questionnaires}
\label{sec:questionnaires}
\newpage
\includepdf[pages={1-}]{appendix/questionnaires/general_2.pdf}
\newpage
\includepdf[pages={1-}]{appendix/questionnaires/with_tools_2.pdf}
\newpage
\includepdf[pages={1-}]{appendix/questionnaires/without_tools_2.pdf}
\newpage
\includepdf[pages={1-}]{appendix/questionnaires/concluding_2.pdf}

\chapter{Tasks}
\label{sec:tasks}
\newpage
\includepdf[pages={1-}]{appendix/tasks/xdyt-bug.pdf}
\newpage
\includepdf[pages={1-}]{appendix/tasks/xdyt-impl.pdf}
\newpage
\includepdf[pages={1-}]{appendix/tasks/xdc-bug.pdf}
\newpage
\includepdf[pages={1-}]{appendix/tasks/xdc-impl.pdf}


\listoffigures

\chapter*{Acknowledgements}

\begin{center}
	\textbf{I would like to thank:}
\end{center}

\textit{Maria Husmann}, for her support, her reviews and her great ideas.

\textit{Prof. Dr. Moira Norrie}, head of the GlobIS research group, for allowing me to write this thesis.

\textit{All participants of the user study}, for their time and effort.

\newpage
\thispagestyle{empty}

\bibliographystyle{plain}
\bibliography{bibliography}

\newpage
\includepdf[pages={1-}]{declaration_of_originality.pdf}

\end{document} 
